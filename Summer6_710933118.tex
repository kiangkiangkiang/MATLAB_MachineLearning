%\input{../Jiang_Preamble}
%\title{\MJHmarker 監督式學習 \\ \Large{\textbf{類神經網路}}}
%\author{{\MJH 江柏學}}
%\date{\today}  
%\begin{document}
%	\maketitle
%	\fontsize{12}{22pt}\selectfont
\chapter{\MJH 監督式學習之類神經網路}
	類神經網路,又稱人工神經網路(Artificial Neural Network,簡稱ANN),在機器學習和認知科學領			域,是一種模仿生物神經網路(動物的中樞神經系統,特別是大腦)的結構和功能的數學模型或計算模型,用		於對函式進行估計或近似。而此概念也能夠透過各種程式語言實現,例如MATLAB、PYTHON,但在實作類神經		網路之前,必須先探討其中之理論,以及運作方式,之後方能進行實務上之運用。
	\section{類神經網路(ANN)}	
		一般神經網路的階層上,包含輸入層、隱藏層、輸出層,而中間的隱藏層可以有一層以上。其中,兩層			(含)以上隱藏層的神經網路,通常會被泛稱為深度神經網路(Deep Neural Network,DNN),如			圖\ref{fig:ann_concept}\ 所示。\footnote{資料來源:https://medium.com/marketingdatascience/\%E5\%BF\%AB\%E9\%80\%9F\%E5\%8F\%8D\%E6\%87\%89\%E6\%A9\%9F\%E5\%88\%B6-\%E9\%A1\%9E\%E7\%A5\%9E\%E7\%B6\%93\%E7\%B6\%B2\%E8\%B7\%AF-a3bbdee4a6f6}
		\begin{figure}[H]	
			\centering	 			 	 
   			\includegraphics[width=0.6\textwidth]{\imgdir ann_concept.jpg} 
   			\caption{類神經網路之概念圖}
   			\label{fig:ann_concept}   			 		 
		\end{figure}
		假設輸入層有$p$個變數,$x_1,x_2,\cdots,x_p$,輸出層有$r$個變數,$y_1,y_2,\cdots,y_r			$類神經網路的概念就是討論輸入層之變數$x$與輸出層之變數$y$之間的非線性函數關係,如式				(\ref{eq:y_fx}):
		\\
		\begin{equation}\label{eq:y_fx}
 				y_k = f_k(x_1,x_2,x_3,\cdots ,x_p)\\
 		\end{equation}
		\\
		由式(\ref{eq:y_fx})\ 可看出,類神經網路認為每項$y$都可以由$x_1,x_2,\cdots,x_p$共同反			應而得,而其中之反應,也是隱藏層內部的非線性函數轉換,換言之,將$x_1,x_2,\cdots,x_p$經由			隱藏層內之函數做非線性組合,便可以得到預測結果$y$,而圖\ref{fig:ann_concept}\ 中,包含			兩個隱藏層,可見其中輸入值經由兩種不同函數作用,方能得到輸出值$y$。
		\\
		\\
		然而,在輸入值進入隱藏層,以及隱藏層準備做輸出之間,還有一項控制因素,權重(weight),考量到			一項輸出值$y$是由$x_1,x_2,\cdots,x_p$共同作用而來,但並不能確保每項$x$對於$y$都有相同			的影響力,因此將每個輸入值$x$做權重調整$w$後才做函數轉換,舉例來說,若要做出一項動作($y				$),是透過數種神經元$x$共同觸發而得,但並不一定每條神經元在傳導(接觸)的效用相同,可能有的			神經元觸發效果大,而有的神經元小,因此才需要權重作調整,如圖\ref{fig:weight}\ 所示。\footnote{圖片來源;http://www.cc.ntu.edu.tw/chinese/epaper/0038/20160920\_ 3805.html}
		\begin{figure}[H]	
			\centering	 			 	 
   			\includegraphics[width=0.5\textwidth]{\imgdir weight.jpg} 
   			\caption{權重調整輸入值}
   			\label{fig:weight}   			 		 
		\end{figure}
		再者,經由函數轉換後的非線性組合,將再加上殘差值(bias)後再傳給下一個神經元,如此便完成了該			節點的輸出,而此節點的輸出,即為下個神經元之輸入值,如此一層一層傳導,直到最後的輸出層,便			產生預測結果,如此方式在學術上,稱為「前向傳播法(Forward-Propagation)」。
		\\
		\\		
		最後,透過前向傳播法,我們得到了預測結果,便可以透過此預測結果和真實值做相減再平方最後取偏			偏微分,如式(\ref{eq:lsd}),透過此類似最小平方法之概念,便可找出能讓殘差最小之權重。
		\begin{equation}\label{eq:lsd}
 				\nabla \sum(y_i - \hat{y_i})^2
 		\end{equation}
		因此,類神經網路可以透過大量的資料集訓練,並且經由複雜的數學計算預測最終結果,而其中訓練資			料便是能影響最後預測結果的關鍵之一,若是訓練資料集具有代表性,且足夠多,最終便會產生較小的			誤差,反之,若是訓練資料瑕疵較多,最終測試結果便會差強人意。
		\bigskip
	\section{ANN實務應用——機器手臂}
		而此類神經模型,在實務也被廣泛應用,例如商業分析、金融預測、圖像辨識以及機器人,都是當前常			會出現的議題,而本文也將淺談機器人中的機器手臂運動,作為類神經之應用,並且將透過MATLAB作為			實作工具,了解其中語法,以及利用圖像理解。
		\\		
		假設在平面空間中,有項手臂運動,其運動範圍之控制因素由手臂角度($\theta_1$)和關節角度($				\theta_2$)所控制,而若有項物體在($x_1,x_2$)點上,手臂要如何改變角度才能觸碰到此物體,也			就是說,假設給定($x_1,x_2$),希望求($\theta_1,\theta_2$)為何,而此問題可以更簡單想成式			(\ref{eq:theta}):
		\begin{equation} 
            \begin{split} \label{eq:theta}
                \theta_1 = f_1(x_1,x_2)\\
                \theta_2 = f_2(x_1,x_2)
            \end{split}
        \end{equation}
        式(\ref{eq:theta})中,可看出在已知($x_1,x_2$)的情況下,透過某種函數轉換,能得出最終結果
        $theta$,而此問題模式,正如同類神經網路所討論,透過函數轉換輸入值,並且經由權重,殘差所調			整後,得出預測值,而要如何判斷此模型好壞,或是如何增加預測正確度,便是從大量資料之訓練學習			而得,因此回到主題,在進行建模之前,需要先有一組訓練資料集,而由上述提到,訓練資料集的多寡			以及是否具有代表性,有可能影響最終預測結果,因此我們在機器手臂可能的運動範圍內,討論以下幾			種模擬資料,來討論訓練資料集之優劣:
        \\
        \begin{enumerate}        	
        	\item {\textbf{樣本數N=100}\\
        		根據上述問題,我們先假設機器手臂能活動的範圍為圖\ref{fig:area}\ 之灰色區塊:
        		\begin{figure}[H]	
					\centering	 			 	 
   					\includegraphics[width=0.5\textwidth]{\imgdir area.jpg} 
   					\caption{機器手臂運動範圍(灰色區域)}
   					\label{fig:area}   			 		 
				\end{figure}
        		那模擬資料之抽樣,亦會在此黑色區塊內做抽樣,而若是以樣本數為100做抽樣,則會如圖					\ref{fig:sample100}\ 所示:
        		\begin{figure}[H]	
					\centering	 			 	 
   					\includegraphics[width=1\textwidth]{\imgdir sample100.jpg} 
   					\caption{機器手臂之訓練資料集(N=100)}
   					\label{fig:sample100}   			 		 
				\end{figure}
				可見資料散佈程度較稀疏,仍有許多運動空間尚未訓練,理論上來說可能無法有良好的預測,					然而,我們仍此資料作為實驗,來觀察最終結果是否符合理論。        	
        	}
        	\item {\textbf{樣本數N=400}\\
        	\\
        		而在對照組的部分,我們以樣本數為400來做測試,由圖\ref{fig:sample400}\ 可見,樣					本數在400時,已經能涵蓋大部分之運動範圍,但在外圍的部分,可能仍有訓練較為不足之					處,因此可能在預測時外圍之誤差會較高,同樣我們列為本次實驗之資料集,以實作來觀察結					果是否和預期一樣,也和另一組資料做比較,討論樣本數多寡時所造成的誤差。
        		\begin{figure}[H]	
					\centering	 			 	 
   					\includegraphics[width=1\textwidth]{\imgdir sample400.jpg} 
   					\caption{機器手臂之訓練資料集(N=400)}
   					\label{fig:sample400}   			 		 
				\end{figure}
        	}
        \end{enumerate}        
        在正式進入實驗之前,必須先設定類神經網路中隱藏層內的層數(q),若層數愈高,其中非線性程度也愈			高,我們在此透過程式來測試不同層數最後的預測成效,而可以先觀察圖\ref{fig:arms_real}\ ,			由於目前要進行為監督式學習,因此在學習前須要先具備已知資料($x_1,x_2,\theta_1 ,					\theta_2$),而透過方程式轉換,我們將模擬資料之($x_1,x_2$)轉換為目標變數($\theta_1,\				theta_2$),如此便有真實值可做監督式學習之建模。
        \\
        \\
        接著再透過MATLAB內建之類神經學習器("Neural Net Fitting")進行建模實驗,而在本次實驗中,			我們除了將測試兩種不同數量之資料集以外,同時測試當隱藏層為10至20時,正確率會有何變化。        
        
       \begin{figure}[H]	
			\centering	 			 	 
   			\includegraphics[width=1\textwidth]{\imgdir arms_real.jpg} 
   			\caption{目標變數之真實值資料}
   			\label{fig:arms_real}   			 		 
		\end{figure}				
        
        而結果如表\ref{table:armsResult}\ 所示:
        \begin{table}[h]				
			\caption{不同隱藏層之performance(N=100)}\label{table:armsResult}
			\centering
			\extrarowheight=8pt
			\begin{tabular}{llllll} 					
				\hline
				L=10\ &L=12 \ &L=14 \ &L=16 \ &L=18 \ &L=20\\ \hline	
				0.0021\ & 0.0028\ & 0.0017\ & 0.0017\ & 0.0013\ & 0.0028\\ \hline	
			\end{tabular}
		\end{table}
		由表\ref{table:armsResult}\ 可看出L=12與20之performance最高,而L=18最低,當隱藏層數			不同時,每個模型表現亦會有差距,而接著我們測試當n=400時,各個模型之表現,結果如表					\ref{table:armsResult2}:
		\begin{table}[H]				
			\caption{不同隱藏層之performance(N=400)}\label{table:armsResult2}
			\centering
			\extrarowheight=8pt
			\begin{tabular}{llllll} 					
				\hline
				L=10\ &L=12 \ &L=14 \ &L=16 \ &L=18 \ &L=20\\ \hline	
				0.0010\ & 0.0011\ & 0.0015\ & 0.0011\ & 0.0014\ & 0.0011\\ \hline	
			\end{tabular}
		\end{table}
		表\ref{table:armsResult2}\ 可見當樣本數大時,大部分performance都下降,可見當樣本數愈			大時,誤差愈小,估計愈準確,而performance最小則是發生在L=10處。
		\\
		\\
		而透過此機器手臂之實驗,我們可以知道樣本數對於預測之重要性,而此例是將資料透過類似系統式抽			樣的方式,每隔一段半徑及一個角度抽取一個樣本,如此抽樣也是希望能夠不遺漏所有母體內之資料,			但倘若以雖機抽樣的方式,均勻分佈在母體上,可能也會有較好的成效。
        
	\section{ANN實務應用——分類器}
		類神經網路亦可以用在分類上,例如已知$x_1,x_2$,預測$y$的類別,而在此透過既有資料"Demo"來			做實作ANN之分類,同樣我們先觀察資料集基本分佈狀況,如圖\ref{fig:demo}:
		\begin{figure}[H]	
			\centering	 			 	 
   			\includegraphics[width=0.8\textwidth]{\imgdir demo.jpg} 
   			\caption{模擬資料集之散佈圖}
   			\label{fig:demo}   			 		 
		\end{figure}
		而透過MATLAB中的"Neural Net Pattern Recognition"便可以直接預測結果,如圖						\ref{fig:demoConfusion}:
		\begin{figure}[H]	
			\centering	 			 	 
   			\includegraphics[width=0.6\textwidth]{\imgdir demoConfusion.jpg} 
   			\caption{confusion matrix}
   			\label{fig:demoConfusion}   			 		 
		\end{figure}
		由圖\ref{fig:demoConfusion}\ 可看出,訓練正確率為93.6\%,而測試正確率為93.3\%,在分類			上已有不錯表現,再者,我們透過不同的隱藏層,觀察資料正確率變化,如表\ref{table:c2Err}:
		\begin{table}[H]				
			\caption{不同隱藏層之錯誤率以兩群為例}\label{table:c2Err}
			\centering
			\extrarowheight=8pt
			\begin{tabular}{llllll} 					
				\hline
				L=10\ &L=12 \ &L=14 \ &L=16 \ &L=18 \ &L=20\\ \hline	
				0.065\ & 0.125\ & 0.060\ & 0.070\ & 0.070\ & 0.060\\ \hline	
			\end{tabular}
		\end{table}
		表\ref{table:c2Err}\ 則顯示當隱藏層愈多時,錯誤率表現不一定愈低,在L=20與L=14有同樣的表			現,而在此例中,L=12反而有最高之錯誤率存在。而除了兩群分類,本文亦討論在三群時ANN之表現如			何。
		\\
		\\
		在三類分群上,我們用相同的方式進行分類,而其中Confusion Matrix如圖\ref{fig:c3Confu}:
		\begin{figure}[H]	
			\centering	 			 	 
   			\includegraphics[width=0.7\textwidth]{\imgdir c3Confu.jpg} 
   			\caption{confusion matrix (三群)}
   			\label{fig:c3Confu}   			 		 
		\end{figure}
		由圖\ref{fig:c3Confu}\ 可看出在三群分類上,訓練正確率高達96\%,而測試正確率也有95\%的高			正確率,可見ANN在三群分類上亦有相當成效,雖然無法像LDA,QDA等等擁有線段繪圖,以輔佐分類理			解,但ANN卻也帶給分類之高效益,最後,我們同樣測試不同隱藏層之表現,如表							\ref{table:c3Err}:
		\begin{table}[h]				
			\caption{不同隱藏層之錯誤率以三群為例}\label{table:c3Err}
			\centering
			\extrarowheight=8pt
			\begin{tabular}{llllll} 					
				\hline
				L=10\ &L=12 \ &L=14 \ &L=16 \ &L=18 \ &L=20\\ \hline	
				0.0387\ & 0.0325\ & 0.0413\ & 0.0413\ & 0.0437\ & 0.0450\\ \hline	
			\end{tabular}
		\end{table}
		表\ref{table:c3Err}\ 可見隨著隱藏層愈多,錯誤率愈高,由於隱藏層愈多會導致非線性成分愈				高,因此就此例而言,有可能在分類上不需要太高之非線性函數,因此在隱藏層數為12時就已經達到高			水準的正確率,而由此二例我們都可以觀察的出來,不同的隱藏層導致不同的錯誤率存在,而隱藏層的			選擇也是ANN中需要考量的重點之一。
	\section{結論}
		在本文中,僅是粗淺的談論類神經網路,若是在數學上討論,會包含更多函數存在,而當隱藏層愈多,			愈接近深度學習,其中變化也愈高,而本文透過簡單的機器手臂預測點,以及分類來說明ANN之用途,並			且討論不同層數之錯誤率成效,而由實驗也可知,樣本數若是無法大致涵蓋母體,將會導致正確率下				降,因此樣本數的多寡以及樣本數分佈情形,對於最終預測結果相當重要,且隱藏層數也須經由測試得			知效果。
%\end{document}















