\chapter{ \MJH MATLAB分類方法總結}
本文歸納了多種不同的分類模型,其中在分類成效上各有優缺,並在理論觀點上也各不相同,有的基於分配假設下建立分類器,有的則是建立在模型假設下,判斷機率高低,也有的分類模型僅在直觀的基礎架構下,做出類別的預測,而本章主要總合所有分類器,一併討論其中優劣。
\section{資料集建立}
在比較所有模型之前,透過MATLAB先建立不同種類的資料集,而在此分別探討3種不同的資料集之型態,並且最後透過以下3種不同資料集,進行各個模型的測試,比較優劣。
	\begin{enumerate}
	\item{\textbf{資料集一 常態分配且變異數相同}\\
	資料集一中,我們建立了常態分配的資料,並且令其變異數矩陣皆相同,最後透過散佈圖顯示如圖\ref{fig:dataset1_scatter};
	\begin{figure}[H]	
		\centering	 			 	 
   		\includegraphics[width=1\textwidth]{\imgdir dataset1_scatter.jpg} 
   		\caption{常態分配且變異數相同之資料集}
   		\label{fig:dataset1_scatter}   			 		 
	\end{figure}
	資料集一,利用最常見之常態分配建立資料,並且作最單純的變異數矩陣相同,測試模型最後的效能如何,觀察是否在假設常態分配下的分類器,會有較優異的表現,或是假設變異數矩陣相同的模型,能展現低誤差。	
\\	
	其中在模擬資料集一之建立下,令$\mu_1 = (3,2),\mu_2 =(4,6)$且變異數矩陣如式(\ref{eq:dataset1_sigma}):
	\begin{equation} \label{eq:dataset1_sigma}
	\sum = \left[
            \begin{array}{cc}
                1 & 0.2 \\
                0.2 & 1    
            \end{array} \right]
    \end{equation}
    可見兩群資料皆有共變異存在,而第二群資料之平均數較第一群高。
	}
	
	\item{\textbf{資料集二 常態分配但變異數不同}\\
接著,在資料集二當中,主要測試在常態分配下,若是變異數矩陣不同,各個分類器會有何表現,因此我們先令$\mu_1 = (3,2),\mu_2 =(4,6)$和資料集一相同,保持不變,但變異數矩陣改成式(\ref{eq:dataset2_sigma}):
	\begin{equation} 
	\begin{split}\label{eq:dataset2_sigma}
	\sum_{1} = \left[
            \begin{array}{cc}
                1 & 0.1 \\
                0.1 & 1    
            \end{array} \right]\\ 
     \sum_{2} = \left[
            \begin{array}{cc}
                8 & 0.4 \\
                0.4 & 3    
            \end{array} \right] 
    \end{split}  
    \end{equation}
	可由式(\ref{eq:dataset2_sigma})\ 看出兩共變異數矩陣有所差別,第一組資料之變異程度較小,而第二組資料相較變異程度較大,且對於$x_1$有較高變異,因此散佈圖可能也會呈現較廣的分佈,如圖\ref{fig:dataset2_scatter};
\begin{figure}[H]	
		\centering	 			 	 
   		\includegraphics[width=1\textwidth]{\imgdir dataset2_scatter.jpg} 
   		\caption{常態分配但變異數不同之資料集}
   		\label{fig:dataset2_scatter}   			 		 
	\end{figure}	
	圖\ref{fig:dataset2_scatter}\ 則呈現資料一較為密集,而資料而如上述所分析,分佈較廣,並且兩資料重疊部分較高,而此資料集主要希望測試是否共變異數矩陣不同時,加廣型迴歸方式能有較好的表現。
	}
	
	\item{\textbf{資料集三 非常態分配}\\
	最後,在資料集三中,我們討論到當兩資料來自不同分配,且都不是常態分配時,模型的表現如何,而在此令第一組資料來自卡方分配,其中透過標準常態分配之平方轉換,第二組資料來自伽馬分配,程式碼如下所式:
	
	\begin{lstlisting}
Ntrain=1000;Ntest=1000;
n1=Ntrain+Ntest;n2=n1;n=n1+n2;
mu1=[0 0];
sigma=[1 0;0 1];
A=mvnrnd(mu1,sigma,n1);
B=gamrnd(4,1.5,[2000 2]);
A=A.^2;
x=[A;B];y=[zeros(n1,1);ones(n2,1)];
gscatter(x(:,1),x(:,2),y,'br','ox');
xlim([0 20]);ylim([0 20]);
index = randperm(n);
D.test.x=x(index(1:Ntest),:); 
D.test.labels=y(index(1:Ntest),:);
D.train.x=x(index(Ntest+1:end),:); 
D.train.labels=y(index(Ntest+1:end),:);	
	\end{lstlisting}
	令第一組資料集為卡方分配,且自由度為一,第二組資料集為伽馬分配,且參數$\alpha,\beta$分別為$(4,1.5)$,最後透過訓練資料集繪製圖形,如圖\ref{fig:dataset3_scatter}:
	\begin{figure}[H]	
		\centering	 			 	 
   		\includegraphics[width=1\textwidth]{\imgdir dataset3_scatter.jpg} 
   		\caption{非常態分配之資料集}
   		\label{fig:dataset3_scatter}   			 		 
	\end{figure}	
	圖\ref{fig:dataset3_scatter}\ 可看出卡方分配散落在座標軸附近,而伽馬分配則是分佈較廣,但兩者仍有較明顯之分類線存在,以此資料集,測試是否建立在常態分配之假設下的分類器,會有較高的錯誤率。
	}
	\end{enumerate}
	
\section{模型測試}		
先前對於分類,預測之模型,總共包含以下幾種,REG、REG*、LDA、QDA、KNN、ANN,其中KNN又可由附近K個值做參數的變動,因此由KNN又可分出KNN(10),KNN(20),總共有7種分類器,而ANN則是以隱含層數為10進行測試比較,以下將分別由三種資料集,進行各個模型之比較:
\begin{enumerate}
\item{\textbf{資料集一:}\\
	資料集一我們了解資料分佈皆為常態,且變異數矩陣相同,以肉眼觀察可能能用直線做分類,因此REG與LDA可能會有較高的表現,透過迴圈,重複100次實驗後取平均值之結果如表\ref{tab:dataset1_result}:
	\begin{table}[H]				
		\caption{所有分類器比較\_ 以資料集一為例}\label{tab:dataset1_result}
		\centering
		\extrarowheight=10pt
		\begin{tabular}{lrrrrrrr} 					
		\hline
正確率   &REG &REG*    &LDA     &QDA     &KNN(10) &KNN(20) &ANN \\ \hline 
訓練& 97.43\% &97.48\% &97.43\% &97.42\% &97.44\% &97.41\% &95.73\%   \\ 	
測試& 98.24\% &98.14\% &98.24\% &98.23\% &98.19\% &98.30\% &92.33\%   \\ 		
		\hline					
		\end{tabular}
	\end{table}
由表\ref{tab:dataset1_result}\ 可見,訓練正確率最高者為加廣型迴歸,而測試正確率最高則是KNN(20),而其中LDA與REG表現為次高者,可能由於資料集一中,既符合常態,也服從共變異數一致之假設,因此有較好的表現,而ANN則是處於墊底狀態,由於ANN較屬於高度非線性模型,因此在分類線類似直線的情形下,ANN成效可能較差,由圖\ref{fig:dataset1_result}\ 更可清楚觀察出差異所在。
\begin{figure}[H]	
		\centering	 			 	 
   		\includegraphics[width=1\textwidth]{\imgdir dataset1_result.jpg} 
   		\caption{所有模型正確率 以資料集一為例}
   		\label{fig:dataset1_result}   			 		 
	\end{figure}
	圖\ref{fig:dataset1_result}\ 可見ANN之正確率相較於其他模型下非常低,可見高度非線性模型並不適用於線性分類上,至於其他模型之正確率則是相去不遠。
}
%接著寫dataset2 或是要把ANN除掉的圖放上去
\item{\textbf{資料集二:}\\
接著,我們以資料集二做實作,資料集二雖也屬於常態,卻是不同共變異數矩陣之情形,同樣以重複100次做實驗,而得到之結果如表\ref{tab:dataset2_result}:
\begin{table}[H]				
		\caption{所有分類器比較\_ 以資料集一為例}\label{tab:dataset2_result}
		\centering
		\extrarowheight=10pt
		\begin{tabular}{lrrrrrrr} 					
		\hline
正確率   &REG &REG*    &LDA     &QDA     &KNN(10) &KNN(20) &ANN \\ \hline 
訓練& 93.23\% &94.76\% &93.23\% &95.69\% &95.17\% &94.90\% &94.04\%   \\ 	
測試& 93.53\% &94.75\% &93.53\% &95.17\% &95.00\% &95.00\% &92.39\%   \\ 		
		\hline					
		\end{tabular}
	\end{table}
	在表\ref{tab:dataset2_result}\ 中,訓練正確率最高為QDA模型,而測試正確率最高同樣也是QDA模型,可見在常態分配以及共變異數矩陣不等時,完全符合QDA的假設前提,因此與理論相符,擁有最好的分類表現,反而LDA相較之下,正確率則是低了不少,由圖\ref{fig:dataset2_result}\ 可見所有模型之間的差異。
	\begin{figure}[H]	
		\centering	 			 	 
   		\includegraphics[width=1\textwidth]{\imgdir dataset2_result.jpg} 
   		\caption{所有模型正確率 以資料集二為例}
   		\label{fig:dataset2_result}   			 		 
	\end{figure}
	而ANN在圖\ref{fig:dataset2_result}\ 同樣能清楚看出,測試時表現不理想,但訓練時的正確率卻不低,可能在訓練模型上,非線性程度過高,導致過度訓練的情形發生,因此才有此現象。
}

\item{\textbf{資料集三:}\\
最後,我們以資料集三做實作,資料集三屬於兩非常態資料集,目測觀察可能分類線一樣是非線性曲線,而由MATLAB執行結果如表\ref{tab:dataset3_result}:
\begin{table}[H]				
		\caption{所有分類器比較\_ 以資料集一為例}\label{tab:dataset3_result}
		\centering
		\extrarowheight=10pt
		\begin{tabular}{lrrrrrrr} 					
		\hline
正確率   &REG &REG*    &LDA     &QDA     &KNN(10) &KNN(20) &ANN \\ \hline 
訓練& 93.67\% &95.86\% &93.67\% &93.76\% &96.05\% &95.63\% &94.58\%   \\ 	
測試& 92.92\% &95.03\% &92.92\% &93.22\% &95.55\% &95.28\% &93.80\%   \\ 		
		\hline					
		\end{tabular}
	\end{table}
	表\ref{tab:dataset3_result}\ 可見,KNN在此範例資料集表現良好,反而同樣建立在常態分配假設下的LDA與QDA相對而言就無法有較高的正確率,而在此兩者皆為非常態模型下,KNN亦能準確預測,並且在前兩資料集中,KNN表現相較之下正確率也偏高,可見在無任何假設下的KNN能對更廣泛的資料集有好的預測,由圖\ref{fig:dataset3_result}\ 也可以明顯看出KNN正確率之高。
	\begin{figure}[H]	
		\centering	 			 	 
   		\includegraphics[width=1\textwidth]{\imgdir dataset3_result.jpg} 
   		\caption{所有模型正確率 以資料集三為例}
   		\label{fig:dataset3_result}   			 		 
	\end{figure}
	而圖\ref{fig:dataset3_result}\ 也可見,加廣型迴歸也有高水準之正確率表現,由於其僅假設模型為非線性模型,因此對於此資料集而言,也較為適合。
}
\end{enumerate}
\newpage
\section{結論}
結合三種資料集之所有模型正確率,可以得到表\ref{tab:all_result}\ 如下所示:
\begin{table}[H]				
		\caption{所有分類器之正確率}\label{tab:all_result}
		\centering
		\extrarowheight=10pt
		\begin{tabular}{l|rr|rr|rr} 
		\hline
\multicolumn{3}{r}{資料集一}	&\multicolumn{2}{c}{資料集二}&\multicolumn{2}{c}{資料集三}\\
		\hline		
   正確率&訓練 & 測試 &訓練 & 測試 &訓練 & 測試  \\ \hline 
REG& 97.43\% &98.24\% &93.23\% &93.53\% &93.67\% & 92.92\% \\ 	
REG*& 97.48\% &98.14\% &94.76\% &94.75\% &95.86\% &95.03\%   \\ 	
LDA& 97.43\% &98.24\% &93.23\% &93.53\% &93.67\% &92.92\%   \\ 
QDA& 97.42\% &98.23\% &95.69\% &95.17\% &93.76\% &93.22\%   \\ 
KNN(10)& 97.44\% &98.19\% &95.17\% &95.00\% &96.05\% &95.55\%   \\ 
KNN(20)& 97.41\% &98.30\% &94.90\% &95.00\% &95.63\% &95.28\%   \\ 
ANN& 95.73\% &92.33\% &94.04\% &92.39\% &94.58\% &93.80\%   \\ 	

 
		\hline					
		\end{tabular}
	\end{table}		
		KNN在所有資料集中的正確率,都在水準之上,而需要建構在假設下的LDA與QDA則是有著起起伏伏的正確率表現,而ANN在此三種資料集中,也不盡理想,可能原因是三種資料集都無法讓ANN展現其高度非線性之模型,而由表\ref{tab:all_result}\ 我們便可對於先前所講到的分類器,有著更深入的認知,在不同資料情形下,分類器之選擇也是極為重要的議題之一,倘若在分類器正確率差異不遠的情形下,也可以選擇成本,效率較好的分類器,如REG或LDA,由此例同時也可證明MATLAB在不同模型進行下,能夠快速進行多次建模預測之處理,並且同時繪圖以讓大眾理解,在進行理論數學探討的同時,以實作圖形輔助,驗證理論並且加強記憶,不僅僅對於分類器之學習,對於其他廣泛議題仍有相當高的助益。
	
		
		
		
		
		
		
		
		
		