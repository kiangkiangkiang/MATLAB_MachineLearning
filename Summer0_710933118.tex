\addcontentsline{toc}{chapter}{序}
\chapter*{序}
對於統計而言,數學是理論的根本,預測與分析則是能善加利用的工具,然而,在這之間許多人已經受理論所馴服,對於工具僅有表面的操作,卻無法熟悉其中的內涵,而往往導致於此之因素皆來自艱困的數學理論,因此本文利用程式剖析數學,利用程式闡發統計,將原本艱澀的數學以及抽象的理論以圖形呈現,並結合程式中的演算法將簡化數學,讓讀者不再對於理論有空泛的想法,並也不再畏懼數學攏長的計算過程。
\\
\\
而其中,MATLAB則是本文所示範的程式軟體,因為其簡便的繪圖能力及內建機器學習的應用程式,讓我們能更便利著手繪圖以及了解機器學習,而MATLAB在程式語法中,架構與大部分語法相似,而繪圖能力也有其語法操作和使用者介面能圖形化操作,因此對於剛入門的使用者來說,學習上並不吃力,並且在資源上除了許多應用程式能供下載外,也有搜尋器可讓不熟悉的使用者進行查詢,結合上述諸多好處,MATLAB可算是最適合本文介紹的軟體之一。
\\
\\
透過MATLAB,我們將實作統計中許多的理論,並以圖形輔佐觀察,亦透過圖形了解分配在不同參數下的呈現樣貌,並且利用程式,我們能快速探討統計中最主要的議題之一,預測,利用程式實現諸多預測方式,並探討不同預測方式的成效優劣,也以圖形繪製其資料散佈與分類線,將過去所熟悉的議題不再紙上談兵,而是透過電腦直接進行資料分析。
\\
\\
最終,我們除了記錄結果以及分析各資料優劣外,更以 \LaTeX 呈現所有研究過程,除了圖形,表格,更著重在數學的呈現,透過\LaTeX 之學習,也能熟悉在排版配置以及文章包羅萬象的變化,結合\LaTeX 與MATLAB,將統計與數學充分的呈現,即是本文最終目標!








