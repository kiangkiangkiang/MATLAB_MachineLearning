%\input{../Jiang_Preamble}
%\title{\MJHmarker 監督式學習 \\ \Large{\textbf{判別式與KNN}}}
%\author{{\MJH 江柏學}}
%\date{\today}  
%\begin{document}
%	\maketitle
%	\fontsize{12}{22pt}\selectfont
\chapter{\MJH 監督式學習之判別式與KNN}
	在監督式學習中,除了能用傳統統計的迴歸方式作為分群的依據之外,\textbf{判別式(Discriminant)}		與\textbf{K-Nearest\ Neighbors(KNN)}都是此學習模式中良好的分類器,而相較於迴歸強硬的規範分		類線的型態,利用判別式的方式更著重在機率的比較,同樣建構在統計的基礎上,此方式是對資料有著分配的		假設,比較新的資料在哪一種類別的機率較高,而讓我們判斷類別機率高低的依據,即是本文的主題之一,判		別式。而在最後,也將討論在監督式學習中常見的演算法之一,KNN,有別於統計的基礎,KNN在不需經由任		何假設的情況下,透過機器學習的概念做基本預測與分類,以下也將逐一探討這幾種分類方式。
	\section{Discriminant}
		在日常生活中,假設我們有一群資料包含自變數($x_1,x_2$)以及應變數($y,\forall y \in \{ 			0,1\}$),其中$y$屬於類別變數,而我們有興趣知道新的一筆資料是屬於哪種類別,亦即有興趣想知			道新資料中,$y$是$0$還是$1$,而基於機率的角度思考,我們更想知道新資料中,$y=0$的機率高,			還是$y=1$的機率高,因此由直觀角度思考,便是求$P(y=0 \mid X=newX)$以及$P(y=1 \mid 				X=newX)$ 其中$newX$為新資料,而我們透過統計中的\textbf{貝式定理}改寫此機率,如式				(\ref{eq:bayes}):
 		\bigskip
		\begin{equation}\label{eq:bayes} 						
 			P(y=0 \mid X=newX) = \frac{P(X=newX \mid y=0)P(y=0)}{P(X=newX)}
 		\end{equation}
 		\\
 		我們將原本需進行比較的機率利用貝式定理改成式(\ref{eq:bayes})的型態後,分母因為兩者都相			同,比較時並不需考量,因此又可改形成式(\ref{eq:bayes_up}):
 		\bigskip
 		\begin{equation}\label{eq:bayes_up} 						
 			P(X=newX \mid y=0)P(y=0)
 		\end{equation}
 		\\
		其中式(\ref{eq:bayes_up})的乘積,前者又可稱為概似函數值($P(X=newX \mid y=0)$),倘若			我們做出一項大膽的假設,即假設資料服從常態分佈,其概似函數值就能透過概似函數求出,而此概似			函數在此例便是二維常態分配,如式(\ref{eq:muNorm}),因此可以透過估計方式求得,而後者				$P(y=0)$亦可從樣本資料分布估計,因此我們便可從出機率值加以比較大小。
 		\bigskip
		\begin{equation}\label{eq:muNorm} 						
 			f_k(x) = \frac{1}{(2 \pi )^{\frac{p}{2} } \begin{vmatrix}
 			\sum_k
 			\end{vmatrix}^{\frac{1}{2}}}
 			e^{-\frac{1}{2} (X-\mu_0)^{T} \sum_k^{-1} (X-\mu_0) }
 		\end{equation}
 		\\
		其中若假設每群間的共變異數矩陣相同,式(\ref{eq:muNorm})中的$\sum_k$便可改寫成$\sum$,			並且透過對數轉換及取極值化簡後,即可得出最終判別式,如式(\ref{eq:disc}):
		\bigskip
		\begin{equation}\label{eq:disc} 						
 			\delta_k(x)= X^{T} {\sum}^{-1} \mu_0 - \frac{1}{2}\mu_0^{T} {\sum} ^{-1} 				\mu_0 +\log{P(y=0)}
 		\end{equation}
 		\\
 		最後,我們令$P(y=0 \mid X=newX)=P(y=1 \mid X=newX)$即可得到分類線,而在此時,因為我們			假設了每群共變異數相同,因此線段呈現直線,稱作\textbf{LDA(Linear Discriminant 					Analysis)},而倘若部基於此假設情形下建模,線段將會呈現曲線狀,稱作\textbf{QDA(Quadrati  		Discriminant Analysis)},而此二分類器皆可以利用MATLAB實作,以下也將一一討論。
 		\subsection{LDA(Linear Discriminant Analysis)}
			在上述我們已經介紹完LDA的數學內涵以及統計性質,還有其基本的假設,而以下將透過MATLAB展				示其在分類上的成效,以及圖形上的呈現,而我們以既有的資料集為例,包含兩變項($x_1,x_2$)				以及類別變數($y, \forall y \in \{ 0,1\}$),資料集名稱為"Demo",作為實作中的模擬資				料集,而實驗步驟如下:			
			\begin{enumerate}[Step 1:]
				\item {\textbf{取得並了解資料集}\\
					在取得資料集時,我們往往會先簡單觀察資料是否來源正常,會不會有過多離群值,資						料大致分佈狀況,以及有無遺失值等等,先檢視資料,前(預)處理資料後,使得進行之						後分析,而在此也不例外,由於我們已知資料為模擬資料,並無缺失值存在,因此接著						透過盒形圖進行簡單觀察資料分佈以及離群值存在多寡,如圖\ref{fig:ldaBox}:
					\begin{figure}[H]	
		 		 		\centering	 			 	 
   				 		\includegraphics[width=1\textwidth]{\imgdir ldaBox.jpg} 
   			 			\caption{資料大致分布}   		
   			 			\label{fig:ldaBox}   			 		 
					\end{figure}
					由圖\ref{fig:ldaBox}可看出並無離群值存在,並且資料中兩變項分佈相去不遠,僅						有些許變異上不同,中位數也大致坐落同樣位置,而資料並無太大問題,因此我們略過						資料前(預)處理,直接進行統計分析。
				}				
				\item {\textbf{求矩陣$\mu$ 與矩陣$\sum$ }\\
					接著,我們需要透過MATLAB程式語法幫助我們求出矩陣$\mu$ 與矩陣$\sum$,而$							\mu$矩陣,便是要分別求出當$y=0$之下,$x_1$和$x_2$的平均,以及$y=1$之下,						兩者的平均,語法如下:					
					\begin{center}\colorbox{slight}{
						\begin{tabular}{p{0.9\textwidth}}
							\MJHmarker{\textbf{\color{darkblue}{MATLAB語法 :}}}\\
							D=load('Demo.txt');\\
							X=D(:,1:2); y=D(:,3);\\
							gscatter(X(:,1),X(:,2),y,'br','<>');\\
							x1=X(y==0,:);x2=X(y==1,:);\\							
							mu1 = mean(x1)';mu2 = mean(x2)'\\
							text(mu1(1),mu1(2),'O','FontSize',20);\\
							text(mu2(1),mu2(2),'O','FontSize',20);\\
						\end{tabular}
					}
					\end{center}
					接著求出平均後,讓其平均值標記於資料之散佈圖上,觀察是否平均合理,如圖							\ref{fig:ldaMu}:
					\begin{figure}[H]	
		 		 		\centering	 			 	 
   				 		\includegraphics[width=1\textwidth]{\imgdir ldaMu.jpg} 
   			 			\caption{平均值坐落位置}   		
   			 			\label{fig:ldaMu}   			 		 
					\end{figure}
					圖\ref{fig:ldaMu}中,平均值為點"O",大致能夠坐落在兩群之中心位置,因此判斷						平均值可能錯誤機會不大,而接著開始計算$\sum$。
					\\	
					\\											
					在前文提及,LDA是基於假設每群變異數矩陣$\sum$皆相等的情形下,所建構之模型,						而此$\sum$在兩群數量相等時,$\sum = \frac{(\sum_A + \sum_B)}{2}$,其中						$A,B$代表兩群分別為群$A$以及群$B$,在此例中,僅需將$y=0$和$y=1$的共變異數						矩陣求出,相加後除$2$即可得到共同$\sum$,而在程式碼上,僅需加入"sigma = 						(cov(x1)+cov(x2))/2",即可求出。					
				}
				\\
				\item {\textbf{建立判別式}\\
					最後,在資料都準備齊全後,即可開始建立式(\ref{eq:disc}),建立判別式後即可得						知當取得新資料時,他的類別歸屬,在此我們舉例新資料為$[0,3]^{T}$,而其在座標						軸上如圖\ref{fig:ldaNewX}:
					\begin{figure}[H]	
		 		 		\centering	 			 	 
   				 		\includegraphics[width=1\textwidth]{\imgdir ldaNewX.jpg} 
   			 			\caption{新資料(0,3)位置}   		
   			 			\label{fig:ldaNewX}   			 		 
					\end{figure}
					圖\ref{fig:ldaNewX},可看出新資料大約落在群$1$之中,而由判別式計算後,我們						得出,落在群$0$的可能性有$-3.2742$單位,遠小於落在群$1$的可能$3.4207$單						位,因此由判別式我們也可得知,此新資料較可能屬於群$1$。其中語法如下:
					\begin{center}\colorbox{slight}{
						\begin{tabular}{p{0.9\textwidth}}
							\MJHmarker{\textbf{\color{darkblue}{MATLAB語法 :}}}\\
							D=load('Demo.txt');X=D(:,1:2); y=D(:,3);\\	
							gscatter(X(:,1),X(:,2),y,'br','<>');\\
							x1=X(y==0,:);x2=X(y==1,:);\\
							mu1 = mean(x1)';mu2 = mean(x2)'\\
							text(0,3,'O','FontSize',20);\\
							sigma = (cov(x1)+cov(x2))/2;x=[0,3]'\\	
							N1=sum(y==0);N2=sum(y==1);\\							
							pi1 = N1/(N1+N2);pi2 = N2/(N1+N2);\\						
							delta0 = 																					x'*inv(sigma)*mu1-0.5*mu1'*inv(sigma)*mu1+log(pi1)\\
							delta1 = 																					x'*inv(sigma)*mu2-0.5*mu2'*inv(sigma)*mu2+log(pi2)\\
						\end{tabular}
					}
					\end{center}
					\bigskip										
					最後,我們也可利用MATLAB,將此判別之分類線繪製於圖中,如圖										\ref{fig:ldaMdl}:
					\begin{figure}[H]	
		 		 		\centering	 			 	 
   				 		\includegraphics[width=1\textwidth]{\imgdir ldaMdl.jpg} 
   			 			\caption{LDA 分類線}   		
   			 			\label{fig:ldaMdl}   			 		 
					\end{figure}
					圖\ref{fig:ldaMdl}中LDA\ Model即是令$P(y=0 \mid X=newX)=P(y=1 							\mid X=newX)$求出之分類線,而透過此分類線,也可以明顯判斷出新資料$(0,3)$屬						於群$1$。
				}
			\end{enumerate} 	
			\bigskip	
 		\subsection{QDA(Quadratic Discriminant Analysis)}
 			然而,在日常生活中有許多資料都不滿足LDA的變異一致性假設,因此對於那些不滿足假設的資					料,LDA模型並不能做良好的分類,而在面對這樣資料的處理上,我們便拿掉變異數一致性的假					設,形成新的分類模型,QDA,而在此模型上,背後理論基礎只建立在資料服從常態分配上,因此				較LDA來說,更能處理較多型態的資料,而本文也將透過上述"Demo"資料集,利用MATLAB做實					作,展示QDA在實驗上的成效,以及程式碼操作的處理,還有最後圖形上的展示。
 			\\
 			\\
 			在實作上,我們已經知道資料大致的分佈模樣,因此直接透過MATLAB語法進行QDA建模,而					MATLAB中,也有對應的語法,不需要向先前透過理論基礎,土法煉鋼的建立模型,而是透						過"fitcdiscr"來完成模型建立,其程式語法如下:
 			\bigskip
 			\begin{center}\colorbox{slight}{
				\begin{tabular}{p{0.9\textwidth}}
							\MJHmarker{\textbf{\color{darkblue}{MATLAB語法 :}}}\\
							D=load('Demo.txt');\\
							n=size(D,1);g=cell(n,1);X=D(:,1:2);\\
							g(D(:,3)==0)={'Group A'};\\
							g(D(:,3)==1)={'Group B'};\\
							gscatter(D(:,1),D(:,2),g,'br','ox');\\					
							QDA = fitcdiscr(X,g,'DiscrimType','quadratic');\\
							k=QDA.Coeffs(1,2).Const;\\
							Q=QDA.Coeffs(1,2).Quadratic\\
							L=QDA.Coeffs(1,2).Linear\\
							f = @(x1,x2) k+L(1)*x1+L(2)																	*x2+Q(1,1)*x1\^2+Q(1,2)*x1*x2+Q(2,2)*x2\^2
							\\
							hold on;fimplicit(f,'LineWidth',3)\\						
							hold off;\\
				\end{tabular}
			}
			\end{center}				
			我們取得資料後,透過'cell'將原本屬於$\{0,1\}$的資料改成'Group A'和'Group B',接著				透過'fitcdiscr'建立模型,其中參數'DiscrimType'設定為'quadratic'也就是代表著目前建				立的模型為QDA模型,最後透過語法,將QDA物件內的各個屬性取出,包									含'Const'、'Quadratic'等,以建立QDA分類線,而最後展示圖形如圖\ref{fig:qdaMdl}:
			\begin{figure}[H]	
		 		\centering	 			 	 
   				\includegraphics[width=1\textwidth]{\imgdir qdaMdl.jpg} 
   			 	\caption{QDA Model}   		
   			 	\label{fig:qdaMdl}   			 		 
			\end{figure}
 			最後,我們也同樣以新資料$[0,3]$作為測試,利用語法'predict'來預測新資料坐落類別,而實				驗結果顯示,新資料屬於群"Group B"也就是和LDA結果相同為群$1$,而圖形顯示也明顯可看出				資料位於"Group B"之位置,如圖\ref{fig:qdaMdlPre}:
 			\begin{figure}[H]	
		 		\centering	 			 	 
   				\includegraphics[width=1\textwidth]{\imgdir qdaMdlPre.jpg} 
   			 	\caption{QDA Model Predict}   		
   			 	\label{fig:qdaMdlPre}   			 		 
			\end{figure}
			圖\ref{fig:qdaMdlPre}\ 中,點"O"即為新資料位置。
			\\
			\\
			而在此二模型中,以"Demo"資料集為例,透過MATLAB計算可知,兩者在訓練資料上都有超過9成				的正確率,其中LDA的方式更是展現了94\%的高正確率,高於QDA\ 0.5\%的水準,因此可以猜測				此資料集變異數可能較一致,因此以LDA有較簡單,且高效率的結果。
	\section{K-Nearest\ Neighbors(KNN)}
		KNN在機器學習中佔有一席重要之地,因為其能夠以直觀的演算概念,呈現高水準的正確程度,其中不需			要有任何假設的基礎,即能完成預測,而此概念即是在某個新資料周圍,找出K個離此新資料最近的此K			已知資料,並計算此K個資料所屬群組,最後將新資料點判定給此K個資料所屬較多的群組,如圖				\ref{fig:knnNewX}\ 所示;
		\begin{figure}[H]	
		 	\centering	 			 	 
   			\includegraphics[width=1\textwidth]{\imgdir knnNewX.jpg} 
   			\caption{KNN Predict}   		
   		 	\label{fig:knnNewX}   			 		 
		\end{figure}
		我們將K設定成10,亦即計算最近的10個點的類別,而此例最近10個點中,最多的類別是群1,因此判定			新資料較有可能是群1。
		\\
		\\
		然而,K設定成10並無任何依據證明是最佳解,也可設定成其他數值,而K要設定多少變成了KNN演算法			的問題之一,而在此資料集中,我們反覆測試k為5、10、15、20等等,發現在K為5時有著最好的訓練			正確率為95\%,而其他正確率分別為93\%、94\%、93.5\%,因此我們以$K=5$繪製KNN之分類線,如			圖\ref{fig:knnMdl};
		\begin{figure}[H]	
		 	\centering	 			 	 
   			\includegraphics[width=1\textwidth]{\imgdir knnMdl.jpg} 
   			\caption{KNN 分類線}   		
   		 	\label{fig:knnMdl}   			 		 
		\end{figure}
		透過輪廓線,將$0,1$區隔開來,其中所使用到的關鍵函數為"contour",是能繪出平面上座標值的輪			廓,而此例刻意將平面每個以0.05為間隔的點做預測,因此將近預測整個平面座標,再將預測結果和每			個點的位置當作參數傳給"contour",如此一來,此函數就知道平面座標每個點的高低(0,1),接著透			過此高低繪製輪廓線,類似等高線,而若是將此預測完結果也以MATLAB繪製出來,圖形將如圖					\ref{fig:knnMeshgrid}\ 所示:
		\begin{figure}[H]	
		 	\centering	 			 	 
   			\includegraphics[width=1\textwidth]{\imgdir knnMeshgrid.jpg} 
   			\caption{KNN Meshgrid}   		
   		 	\label{fig:knnMeshgrid}   			 		 
		\end{figure}
		其中,程式碼如下:		
 			\begin{center}\colorbox{slight}{
				\begin{tabular}{p{0.9\textwidth}}
							\MJHmarker{\textbf{\color{darkblue}{MATLAB語法 :}}}\\
							figure,hold on;\\
							D=load('Demo.txt');\\
							knn5 = fitcknn(D(:,1:2),D(:,3),'NumNeighbors',5);\\
							gscatter(D(:,1),D(:,2),D(:,3),'br','ox',10)\\
							
							[matrix1,matrix2] = meshgrid(-2:0.05:10,-1:0.05:5);\\
							vec1 = matrix1(:);vec2 = matrix2(:);\\
							m = predict(knn5,[vec1,vec2]);[mm,nn]=size(matrix1);\\
							z=reshape(m,mm,nn);gscatter(vec1,vec2,m,'br','..');\\
							
							contour(matrix1,matrix2,z,[0.5 0.5],'LineWidth',4)\\
							title('KNN Model');grid;legend('0','1','KNN Model');\\
							set(gca,'fontsize',20);hold off;\\
							
				\end{tabular}
			}
			\end{center}
		而此亦可以用立體圖來顯示此預測結果,僅須加上"mesh(matrix1,matrix2,z)"即可完成立體				圖形,如圖\ref{fig:knnMesh}:
		\begin{figure}[H]	
		 	\centering	 			 	 
   			\includegraphics[width=1\textwidth]{\imgdir knnMesh.jpg} 
   			\caption{KNN 立體圖}   		
   		 	\label{fig:knnMesh}   			 		 
		\end{figure}
		圖\ref{fig:knnMesh}\ 中,透過立體圖形,可以更明顯看出來兩群間的差異,利用z座標,將群1撐			高,而群0仍維持平面。
	\section{模型比較}
		綜合以上三種不同的模型,我們對於分類器已有相當程度的了解,但哪種分類器能夠有較好的成效,以			下我們將做測試,比較LDA,QDA,KNN(5),KNN(15),其中KNN的K數量選擇,暫時先依據上述例子的			正確率較高兩者決定,因此先由5與15建模。
		\\
		再者,在資料選擇中,我們模擬出以下兩種資料,如圖\ref{fig:cmpData12}:
		\begin{figure}[H]
    		\centering      			 
       		\subfloat[兩群資料]{
       		\includegraphics[scale=0.15]{\imgdir Class2Data.jpg}}
       		\subfloat[三群資料]{
       		\includegraphics[scale=0.15]{\imgdir Class3Data.jpg}}
       		\caption{實驗資料集}   
   			\label{fig:cmpData12}
		\end{figure}
		圖\ref{fig:cmpData12}\ 中,包含兩種類別的資料,資料筆數為1000筆,以及三種類別的資料,資			料筆數為900筆,分別用來測試模型成效,而在資料備妥後,先以兩種類別資料做實驗。
		\bigskip		
		\begin{enumerate}
		\item {\textbf{兩種類別實驗}\\
			在兩種類別資料實驗中,我們將資料讀入後,隨機將資料區分為訓練資料集和測試資料集,其中兩				者比例為$7:3$,利用70\%的資料訓練模型,其餘30\%的資料測試模型正確與否,程式如下所					示:			
 			\begin{center}\colorbox{slight}{
				\begin{tabular}{p{0.9\textwidth}}
					\MJHmarker{\textbf{\color{darkblue}{MATLAB語法 :}}}\\
					load Class2Data.mat;\\
					n=size(x,1)\\
					p=0.7;\\
					trainNum = n*p;\\
					testNum = n - trainNum;\\
					index = randperm(n);\\
					trainX = x(index(1:trainNum),:)\\
					trainG = y(index(1:trainNum),:);\\
					testX = x(index(trainNum+1:end),:);\\
					testG = y(index(trainNum+1:end),:)\\
				\end{tabular}
			}
			\end{center}
			\bigskip	
			如此一來,訓練資料集便是變數"trainX"和"trainG",其中"G"代表群組別,接著我們透過					MATLAB內建函數,建立LDA、QDA、KNN模型,並且計算其模型之正確率,而建模與計算正確率之				語法如下;
			\begin{center}\colorbox{slight}{
				\begin{tabular}{p{0.9\textwidth}}
					\MJHmarker{\textbf{\color{darkblue}{MATLAB語法 :}}}\\
					LDA = fitcdiscr(trainX,trainG);\\
					QDA = fitcdiscr(trainX,trainG,'DiscrimType','quadratic');\\
					knn5 = fitcknn(trainX,trainG,'NumNeighbors',5);\\
					knn15 = fitcknn(trainX,trainG,'NumNeighbors',15);\\
					trainAccLDA=1-resubLoss(LDA);trainAccknn5=1-resubLoss(knn5)\\
					trainAccQDA=1-resubLoss(QDA);trainAccknn15=1-resubLoss(knn15)					\end{tabular}
			}
			\end{center}
			其中"trainAcc"開頭變數,即是四種模型之訓練正確率,即"training accuracy",且實驗結				果如表\ref{table:c2trainAcc}:
			\bigskip			
			\begin{table}[h]				
				\caption{兩種類別之訓練正確率}\label{table:c2trainAcc}
				\centering
				\extrarowheight=8pt
				\begin{tabular}{p{3cm} p{3cm} p{3cm} p{2cm} } 					
				\hline
				LDA\    &QDA \    &KNN(5) \ &KNN(15) \\ \hline	
				0.9275\ & 0.9612\ & 0.9637\ & 0.9587 \\
				\hline					
				\end{tabular}
			\end{table}
			\bigskip
			表\ref{table:c2trainAcc}\ 可以看出,KNN(5)的訓練正確率最高,而LDA之訓練正確率最				低,但這僅僅只是訓練資料,不一定代表此四種模型最後的表現,因此,我們用測試資料集再做一				次預測,結果呈現如表\ref{table:c2testAcc};
			\bigskip
			\begin{table}[h]				
				\caption{兩種類別之測試正確率}\label{table:c2testAcc}
				\centering
				\extrarowheight=8pt
				\begin{tabular}{p{3cm} p{3cm} p{3cm} p{2cm} } 					
				\hline
				LDA\    &QDA \    &KNN(5) \ &KNN(15) \\ \hline	
				0.9350\ & 0.9700\ & 0.9650\ & 0.9650 \\
				\hline					
				\end{tabular}
			\end{table}
			\bigskip
			從表\ref{table:c2testAcc}可以看出測試資料結果,QDA為正確率最高之模型,其次是KNN,				其中KNN之K為5和15在此次範例並無差異,最後表現最差的為LDA,而我們觀察函數圖形,如圖					\ref{fig:c2AllMdl};
			\begin{figure}[H]	
		 		\centering	 			 	 
   				\includegraphics[width=1\textwidth]{\imgdir c2AllMdl.jpg} 
   				\caption{四種分類模型}   		
   		 		\label{fig:c2AllMdl}   			 		 
			\end{figure}
			圖\ref{fig:c2AllMdl}\ 可以非常明顯看出資料散佈程度差異非常大,不符合變異數一致					性之假設,因此LDA在此資料集表現非常差,而QDA的曲線完美配適此資料分佈位置,最後KNN以不				規則形狀做分類,雖然也能大致區分不同類別,但在表現上較不如QDA,而最終繪圖之程式碼如下				所示:
			\begin{center}\colorbox{slight}{
				\begin{tabular}{p{0.9\textwidth}}
					\MJHmarker{\textbf{\color{darkblue}{MATLAB語法 :}}}\\
					figure,hold on;gscatter(x(:,1),x(:,2),y,'br','ox',10);\\
					k=QDA.Coeffs(1,2).Const;L=QDA.Coeffs(1,2).Linear;\\
					Q=QDA.Coeffs(1,2).Quadratic;\\					
					f=@(x1,x2)k+L(1)*x1+L(2)*x2+Q(1,1)*x1\^2+(Q(1,2)+Q(2,1))*x1*x2+Q(2,2)*x2\^2\\
					fimplicit(f,'LineWidth',4);\\

					k=LDA.Coeffs(1,2).Const;L=LDA.Coeffs(1,2).Linear;\\					
					f=@(x1,x2)k+L(1)*x1+L(2)*x2;\\
					fimplicit(f,'LineWidth',4);[matrix1,matrix2]=												meshgrid(-6:10,-8:10);\\

					
					vec1 = matrix1(:);vec2 = matrix2(:);\\					
					m = predict(knn5,[vec1,vec2]);[mm,nn]=size(matrix1);\\	
					z=reshape(m,mm,nn);\\
					contour(matrix1,matrix2,z,[0.5 0.5],'LineWidth',4,'color','g');\\

					m = predict(knn15,[vec1,vec2]);\\
					z=reshape(m,mm,nn);\\
					contour(matrix1,matrix2,z,[0.5 0.5],'LineWidth',											4,'color','black');\\

					hold off;legend('0','1','QDA','LDA','KNN5','KNN15')\\	
					title('四種分類模型');grid;set(gca,'fontsize',30);\\	
				\end{tabular}
			}
			\end{center}
			\bigskip
			而將資料分割,建模,繪圖三種程式碼彙整,即是本文所實驗之完整程式碼。
				
		}
		\item {\textbf{三種類別實驗}\\
			在三種類別之實驗,我們和先前一樣,隨機將資料分割成訓練資料集與測試資料集,接著進行建模				與計算訓練正確率,如表\ref{table:c3trainAcc}:
			\bigskip			
			\begin{table}[h]				
				\caption{三種類別之訓練正確率}\label{table:c3trainAcc}
				\centering
				\extrarowheight=8pt
				\begin{tabular}{p{2.5cm} p{2.5cm} p{2.5cm} p{2cm} } 					
				\hline
				LDA\    &QDA \    &KNN(5) \ &KNN(15) \\ \hline	
				0.9196\ & 0.9643\ & 0.9625\ & 0.9589 \\
				\hline					
				\end{tabular}
			\end{table}
			\bigskip
			表\ref{table:c3trainAcc}\ 可見此次模型中,訓練正確率最高的為QDA模型,其次是KNN,				而表現最差的依舊是LDA,由圖\ref{fig:cmpData12}\ (b)可見,三種類別的模型,散佈程度				依舊差距極大,應該理論上來說,LDA確實在此依舊不具備良好分類器之條件,而KNN在不基於任				何假設下,也維持高水準之正確率。\\
			\\
			再者,我們一樣透過語法檢視模型測試正確率,如表\ref{table:c3testAcc}:
			\bigskip			
			\begin{table}[h]				
				\caption{三種類別之測試正確率}\label{table:c3testAcc}
				\centering
				\extrarowheight=8pt
				\begin{tabular}{p{2.5cm} p{2.5cm} p{2.5cm} p{2cm} } 					
				\hline
				LDA\    &QDA \    &KNN(5) \ &KNN(15) \\ \hline	
				0.9167\ & 0.9625\ & 0.9500\ & 0.9583 \\
				\hline					
				\end{tabular}
			\end{table}
			\bigskip
			由表\ref{table:c3testAcc}之測試正確率可見,QDA在三種類別之分群上,表現最為優異,而				值得討論的是,KNN(5)在訓練資料時以0.9625之正確率高於KNN(15),但在測試時,卻是由					KNN(15)以0.9583之正確率優過KNN(5),可見在訓練資料時,正確率高不一定能反映在測試時,				若是訓練資料正確率極高,卻在測試時有差強人意的表現,很可能其中有									\textbf{overfitting}之問題存在。\\
			\\
			而overfitting顧名思義就是過度學習訓練資料,變得無法順利去預測或分辨不是在訓練資料內的				其他資料,然而,機器學習的目標就是要訓練機器擁有人類的思考,並且擁有解決一般問題的能				力,即使看到沒有包含在訓練資料的資料,也是要可以正確辨識的,因此,回到實驗主題中,					KNN(5)雖然擁有良好的訓練正確率,但若與KNN(15)比較,我們依舊偏好有著較高測試正確率之				KNN(15)。
				
		}
		\end{enumerate}
	\section{結論}
		綜合以上幾種實驗,我們可以得到表\ref{table:result}\ 之彙整資料:
		\bigskip			
			\begin{table}[H]				
				\caption{綜合兩群及三群之測試正確率}\label{table:result}
				\centering
				\extrarowheight=8pt
				\begin{tabular}{lrrrr} 					
				\hline
				&\multicolumn{2}{c}{兩群實驗}	 &\multicolumn{2}{c}{三群實驗}\\ 									\hline
				    &訓練正確率 &測試正確率 &訓練正確率 &測試正確率 \\ \hline 
	LDA & 92.8\%  &93.5\%    &92.0\%  &91.7\%   \\ 	
	QDA & 96.1\%  & 97.0\%  &96.4\% &96.3\% \\ 
	KNN(5) & 96.4\%  &96.5\% &96.3\%   &95.0\%   \\ 
	KNN(15) & 95.9\%  &96.5\%  &95.9\% &95.8\% \\ 
				\hline					
				\end{tabular}
			\end{table}
			\bigskip
		表\ref{table:result}\ 中QDA在兩實驗都扮演最優異之分類器,因為其高測試正確率,在分類成效上良好,但QDA在訓練資料上並非皆為最佳,在兩群中我們可以看到KNN(5)在訓練上表現較為優異,而KNN分類表現雖然在測試上都並非頂尖,但也一直扮演優良的分類角色,由於其概念簡單,正確率上又不亞於QDA許多,因此在實務上也能占據重要地位。至於LDA在此次實驗中皆未有良好分類表現,由於此次實驗之模擬資料,變異程度相差甚高,因此LDA在理論上已經無法滿足,在實務呈現理所當然也會成效不佳,這也驗證了LDA確實是需要基於各個群組共變異數皆為一致,LDA方能有所表現。\\
	
%\end{document}