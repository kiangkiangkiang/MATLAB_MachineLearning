%\input{../Jiang_Preamble}
%\title{\MJHmarker MATLAB實作函數圖形}
%\author{{\MJH 江柏學}}
%\date{\today}  
%\begin{document}
%	\maketitle
%	\fontsize{12}{22pt}\selectfont
\chapter{\MJH MATLAB實作函數圖形}	
	\textbf{MATLAB}是MATrix LABoratory(矩陣實驗室)的縮寫,是一款由美國The MathWorks公司出		品的商業數學軟體。MATLAB是一種用於演算法開發、資料視覺化、資料分析以及數值計算的進階技術計算語		言和互動式環境。除了矩陣運算、繪製函數/資料圖像等常用功能外,MATLAB還可以用來建立使用者介面及與		呼叫其它語言(包括C、C++、Java、Python和FORTRAN)編寫的程式。

	儘管MATLAB主要用於數值運算,但利用為數眾多的附加工具箱\emph{(Toolbox)}它也適合不同領域的應		用,例如控制系統設計與分析、圖像處理、訊號處理與通訊、金融建模和分析等。另外還有一個配套軟體包		\textbf{Simulink},提供一個視覺化開發環境,常用於系統類比、動態/嵌入式系統開發等方面。			\footnote{資料源自:維基百科 (https://zh.wikipedia.org/wiki/MATLAB)}
	
	而本文將介紹如何實作MATLAB程式,並且著重在數學之函數圖形上的展現,除了將探討每個數學式子所呈現		的圖形外,更藉由\XeLaTeX 編制,結合兩種方便並廣為人知的語言,介紹數學最原始的樣貌,而以下將先		探討基礎函數圖形,接著深入分析複雜之函數圖形以及利用MATLAB中不同繪圖方式,展現較多元的圖片風			貌。
	
	另外,由於經費限制,硬體設備無法更進,知識涵養也仍在努力充實,EPS圖無法有效顯示,因此以下圖片皆		以JPG圖檔為例,還請讀者多多包容,感謝。
	
	\section{基礎函數圖形介紹 - 以12種函數為例}
		\subsection{$ y=f(x)=sin(x)+cos(x)$}
		\rule{\textwidth}{0.2pt}				
			\begin{center}\colorbox{slight}{
				\begin{tabular}{p{0.9\textwidth}}
					\MJHmarker{\textbf{MATLAB語法 :}}\\					
					f = @(x) sin(x)+cos(x);\\
					fplot(f,[-10,10],'LineWidth',3);\\
					ylim([-2,2]);\\
					grid;\\
					title('f(x) = sin(x) + cos(x)');\\					
				\end{tabular}
			}
			\end{center}			
		\bigskip
		函數圖形呈現如下:
		\begin{figure}[H]	
		 	 \centering	 			 	 
   			 \includegraphics[width=1\textwidth]{\imgdir homework2_plot1.jpg} 
   			 \caption{$ y=f(x)=sin(x)+cos(x)$}   		
   			 \label{plot1}   			 		 
		\end{figure}		
		可以從圖\ref{plot1} 看出 $sin(x)+cos(x)$ 在$y=-1.5$ 與$y=1.5$之間徘迴,形成波型圖,			且圖中也可顯而易見此通過$(0,1)$且至高點與低點在$y=1.414$與$y=-1.414$,而此標示僅需點擊			MATLAB產生之圖片即可,而程式碼中也透漏可以從"LineWidth"調整線的粗細,此題則以3單位為例,			最後以"grid"加上格線,title產生標題,則形成基本函數圖型。		
		\subsection{$y=f(x)=\frac{1-e^{-2x}}{1+e^{-2x}}$}
		\rule{\textwidth}{0.2pt}				
			\begin{center}\colorbox{slight}{
				\begin{tabular}{p{0.9\textwidth}}
					\MJHmarker{\textbf{MATLAB語法 :}}\\					
					f = @(x) (1-exp(-2*x))*(1+exp(-2*x)).$\^$(-1);\\
					plot2 = fplot(f);\\
					xlim([-3,3]);grid;\\					
					title('f(x)=(1-e$\backslash\^$(-2x))/(1+e$\backslash\^$(-2x))')\\
					set(plot2,'linewidth',4);set(plot2,'color','red');\\				
					text(0.1,0,'when x = 0 -> y = 0');\\
					text(1.8,0.92,'lim(x->∞) , y->1');\\
					text(-2.5,-0.9,'lim(x->-∞) , y->-1');\\
					set(plot2,'linestyle',':')\\		
				\end{tabular}
			}
			\end{center}	
		\newpage
		函數圖形呈現如下:
		\begin{figure}[H]	
		 	 \centering	 			 	 
   			 \includegraphics[width=1\textwidth]{\imgdir homework2_plot2.jpg} 
   			 \caption{$y=f(x)=\frac{1-e^{-2x}}{1+e^{-2x}}$}   		
   			 \label{plot2}   			 		 
		\end{figure}
		
		由圖\ref{plot2} 可見,此圖不但通過原點,並且當$x$趨近於負無限大時,$y$將收斂到$-1$,而			當$x$趨近於無	限大時,$y$則趨近於$1$,而從MATLAB角度來觀察,則可發現此圖增加圖片敘述,				以"text"呈現,而線也藉由"linestyle",由實線變成虛線,最後以"color"改變色彩,則可形成較			圖\ref{plot1}更精美的圖型,而由於MATLAB中變數較常以向量形式表現,因此在除法上不易實現,			本例特別由基本除法,改成$-1$次方。
		\newpage
		\subsection{$y=f(x)=\sqrt[3]{\frac{4-x^{3}}{1+x^{2}}}$}
		\rule{\textwidth}{0.2pt}
		\begin{center}\colorbox{slight}{
				\begin{tabular}{p{0.9\textwidth}}
					\MJHmarker{\textbf{MATLAB語法 :}}\\					
					f = @(x) ((4-x$\^$3)*(1+x$\^$2).$\^$(-1))$\^$(1/3);\\
					myFigure = figure;ylim([0,3]);\\
					plot3 = fplot(f,[-20,3]);\\	
					set(plot3,'color','\# ffb5ff');\\					
					title('f(x) = ((4-x$\backslash\^$3)/(1+x$\backslash\^$2))$								\backslash\^$(1/3)')\\
					set(plot3, 'Marker', 'd');\\
					magnify(myFigure);grid;set(plot3,'linewidth',2);\\
				\end{tabular}
			}
			\end{center}			
		\bigskip
		函數圖形呈現如下:	
		\begin{figure}[H]	
		 	 \centering	 			 	 
   			 \includegraphics[width=1\textwidth]{\imgdir homework2_OrginalPlot3.jpg} 
   			 \caption{$y=f(x)=\sqrt[3]{\frac{4-x^{3}}{1+x^{2}}}$}   		
   			 \label{plot3}   			 		 
		\end{figure}
		\bigskip	
		圖\ref{plot3} 中新增了"Marker",讓圖形內每個點更能顯而易見的看出,並且,我們可以看出在			$x=0$時,圖形上有明顯的起伏變化,因此利用額外下載之函數"magnify",讓使用者能透過游標的移			動,有著放大鏡的效果,如此可以更簡易的觀察有興趣之區域,如下圖所示:
		\begin{figure}[H]	
		 	 \centering	 			 	 
   			 \includegraphics[width=1\textwidth]{\imgdir homework2_plot3.jpg} 
   			 \caption{部分放大 $y=f(x)=\sqrt[3]{\frac{4-x^{3}}{1+x^{2}}}$}   		
   			 \label{plot3_zoom}   			 		 
		\end{figure}
		\bigskip
		\subsection{$y=f(x)=\frac{1}{x-1}$}%4
		\rule{\textwidth}{0.2pt}
		\begin{center}\colorbox{slight}{
				\begin{tabular}{p{0.9\textwidth}}
					\MJHmarker{\textbf{MATLAB語法 :}}\\					
					f= @(x) 1/(x-1);\\
					plot4 = fplot(f);\\
					xlim([-1,3]);\\
					ylim([-50,50]);\\
					set(plot4, 'linestyle', ':'); \\
					set(plot4,'linewidth',3);\\
					grid;\\
					title('f(x)=1/(1-x)');\\
				\end{tabular}
			}
			\end{center}			
		\bigskip
		函數圖形呈現如下:	
		\begin{figure}[H]	
		 	 \centering	 			 	 
   			 \includegraphics[width=1\textwidth]{\imgdir homework2_OrginalPlot4.jpg} 
   			 \caption{$y=f(x)=\frac{1}{x-1}$}   		
   			 \label{plot4}   			 		 
		\end{figure}
		\bigskip
		圖\ref{plot4}以原點對稱,並且無限趨近於$x=0$與$y=1$,而此例隨無透過原始程式碼作圖形上的			更動,但卻能由MATLAB內建繪圖程式,對圖片進行修改,以此為例,修改了"grid"之色彩,以及其透			明度,而以下兩圖則是為了能夠更明顯看出漸進效果而增添:
		\begin{figure}[H]
    		 \centering
      		 \subfloat[圖形左側分佈]{
       		 \includegraphics[scale=0.15]{\imgdir homework2_plot4_down.jpg}}
        	 \subfloat[圖形右側分佈]{
       		 \includegraphics[scale=0.15]{\imgdir homework2_plot4_up.jpg}}
   			 \caption{$y=f(x)=\frac{1}{x-1}$}
   			 \label{plot4_updown}
		\end{figure}
		\newpage
		\subsection{$y=f(x)=\frac{1}{2\sqrt{2\pi}}e^{\frac{-(x-1)^2}{8}}$}%5
		\rule{\textwidth}{0.2pt}
		\begin{center}\colorbox{slight}{
				\begin{tabular}{p{0.9\textwidth}}
					\MJHmarker{\textbf{MATLAB語法 :}}\\					
					f = @(x) (1/(2*(sqrt(2*pi))))*exp((-(x-1)$\^$2)/(8));\\
					plot5 = fplot(f);\\
					xlim([-6,8]);\\
					ylim([0,0.3]);\\
					set(plot5,'linewidth',2);\\
					set(plot5, 'MarkerSize', 10); \\
					set(plot5,'color','blue');\\
					grid;\\
					line([1 1],[-100 0.2],'LineStyle',':','linewidth',1,'color','red')
					\\line([-3 -3],[-100 100],'LineStyle',':','linewidth',									1,'color','red')\\
					line([5 5],[-100 100],'LineStyle',':','linewidth',1,'color','red')
					\\text(1,0.21,'mean = 1');\\
					text(-4.3,0.03,'mean - 2sigma');\\
					text(5.1,0.028,'mean + 2sigma');\\
					title('f(x)=(1/2*(2$\backslash$pi)$\backslash\^$(1/2))*e$								\backslash\^$(-(x-1)$\backslash\^$2/8)');\\
				\end{tabular}
			}
			\end{center}			
		\newpage
		函數圖形呈現如下:	
		\begin{figure}[H]	
		 	 \centering	 			 	 
   			 \includegraphics[width=1\textwidth]{\imgdir homework2_OrginalPlot5.jpg} 
   			 \caption{$y=f(x)=\frac{1}{2\sqrt{2\pi}}e^{\frac{-(x-1)^2}{8}}$}   		
   			 \label{plot5}   			 		 
		\end{figure}
		
		圖\ref{plot5}在統計上呈現常態分佈,且平均數為$1$,變異數為$4$,因此才以$x=1$呈現左右對			稱之情形,而此例中更是增加三條虛線,分別是$\mu - 2\sigma$,$\mu$,$\mu+2\sigma$,在				MATLAB語法上則是透過"line"來實現垂直線的做法,其中本例第一個參數是$x$之範圍,第二個參數是			$y$範圍,其後等同"plot"可改變屬性,其中若無法直覺觀察出本例是常態分配的話,可用$x$與$y$界			限調整,讓"xlim"調至令圖形對稱位置,可較輕鬆理解圖形分佈。
		\newpage
		\subsection{$y=f(x)=\sqrt[3]{x^2}$}%6		
		\begin{center}\colorbox{slight}{
				\begin{tabular}{p{0.9\textwidth}}
					\MJHmarker{\textbf{MATLAB語法 :}}\\					
					f = @(x) x$\^$(2/3)\\
					plot6 = fplot(f);\\
					xlim([-10,10]);\\
					ylim([-10,10]);\\
					set(plot6,'linewidth',3);\\
					set(plot6, 'MarkerSize', 8);\\ 
					set(plot6,'color','red');\\
					set(plot6,'MarkerEdgeColor','b');\\
					set(plot6,'MarkerFaceColor','y');\\
					set(plot6, 'Marker', 'd');\\
					title("f(x)=x$\backslash\^$(2/3)");\\
					grid
				\end{tabular}
			}
			\end{center}	
		函數圖形呈現如下:	
		\begin{figure}[H]	
		 	 \centering	 			 	 
   			 \includegraphics[width=1\textwidth]{\imgdir homework2_plot6.jpg} 
   			 \caption{$y=f(x)=\sqrt[3]{x^2}$}   		
   			 \label{plot6}   			 		 
		\end{figure}
		\bigskip
		此例在函數呈現上較單純,由於$f(x)$恆正,因此圖形僅在第一象限呈現,而值得一提的是,本例特別			將"marker"做多方面修改,第一是改變樣式,在圖\ref{plot3}中有提到,而這裡更進一步修改其大			小"MarkerSize",邊線顏色"MarkerEdgeColor",以及其內部填滿顏色"MarkerFaceColor",可			透過不同英文字母,所更動其中各種樣式,最後,背景則透過MATLAB內建繪圖工具完成。
		
		\subsection{$y=f(x)=2x^3-x^4$}%7
		\rule{\textwidth}{0.2pt}
		\begin{center}\colorbox{slight}{
				\begin{tabular}{p{0.9\textwidth}}
					\MJHmarker{\textbf{MATLAB語法 :}}\\					
					f = @(x) 2*x.$\^$(3)-x.$\^$(4);\\
					plot7 = fplot(f);\\
					ylim([-1000,100]);\\
					xlim([-5,6]);\\
					set(plot7,'linewidth',2);\\
					title("f(x)=2x$\backslash\^$3-x$\backslash\^$4");\\
					line([-10 10] ,[0 0],'color','red');\\
					grid
				\end{tabular}
			}
			\end{center}			
		
		函數圖形呈現如下:
		\begin{figure}[H]	
		 	 \centering	 			 	 
   			 \includegraphics[width=1\textwidth]{\imgdir homework2_OrginalPlot7.jpg} 
   			 \caption{$y=f(x)=2x^3-x^4$}   		
   			 \label{plot7}   			 		 
		\end{figure}
		
		圖\ref{plot7} 透過"ylim"的調整縮小後,可以明顯觀察出是一類似拋物曲線,且開口向下,但由此			圖無法看出是否與0有切線,亦或是割線產生,因此藉由以下三種放大可進一步觀察;
		\begin{figure}[H]	
		 	 \centering	 			 	 
   			 \includegraphics[width=1\textwidth]{\imgdir homework2_plot7Left.jpg} 
   			 \caption{函數$y=f(x)=2x^3-x^4$之左側放大}   		
   			 \label{plot7Left}   			 		 
		\end{figure}
		\begin{figure}[H]	
		 	 \centering	 			 	 
   			 \includegraphics[width=1\textwidth]{\imgdir homework2_plot7Mid.jpg} 
   			 \caption{函數$y=f(x)=2x^3-x^4$之中間放大}   		
   			 \label{plot7Mid}   			 		 
		\end{figure}
		\begin{figure}[H]	
		 	 \centering	 			 	 
   			 \includegraphics[width=1\textwidth]{\imgdir homework2_plot7Right.jpg} 
   			 \caption{函數$y=f(x)=2x^3-x^4$之右側放大}   		
   			 \label{plot7Right}   			 		 
		\end{figure}
		其中在圖\ref{plot7Mid}(中圖)中,在放大若干倍可見函數以落在紅線($y=0$)之上,因此更可確定			此函數與$x$軸焦於兩點。
		\newpage
		\subsection{$y=f(x)=x\sqrt{4-x^2}$}%8
		\rule{\textwidth}{0.2pt}
		\begin{center}\colorbox{slight}{
				\begin{tabular}{p{0.9\textwidth}}
					\MJHmarker{\textbf{MATLAB語法 :}}\\					
					f = @(x) x*sqrt(4-x.$\^$(2));\\
					plot8 = fplot(f);\\
					ylim([-2.5,2.5]);\\
					xlim([-3,3]);\\
					set(plot8,'linewidth',2);\\
					title("f(x)=x*sqrt(4-x$\backslash\^$2)");\\
					line([-10 10] ,[2 2],'linestyle',':','color','red');\\
					line([-10 10] ,[-2 -2],'linestyle',':','color','red');\\
					grid\\
				\end{tabular}
			}
			\end{center}			
		
		函數圖形呈現如下:
		\begin{figure}[H]	
		 	 \centering	 			 	 
   			 \includegraphics[width=1\textwidth]{\imgdir homework2_OrginalPlot8.jpg} 
   			 \caption{$y=f(x)=x\sqrt{4-x^2}$}   		
   			 \label{plot8}   			 		 
		\end{figure}
		
		此例是一個類似S形的圖形,其中$x,y$皆介於$2$至$-2$之間,因此刻意加上"line"指令做出虛線,			更方便辨識其定義域,而此例也特別放大其漸進$x=2$與$y=2$的部分,以及$x=-2$和$y=-2$,如下			圖所示:
		
		\begin{figure}[H]
    		 \centering
      		 \subfloat[圖形左側分佈]{
       		 \includegraphics[scale=0.15]{\imgdir homework2_plot8Left.jpg}}
        	 \subfloat[圖形右側分佈]{
       		 \includegraphics[scale=0.15]{\imgdir homework2_plot8Right.jpg}}
   			 \caption{$y=f(x)=x\sqrt{4-x^2}$}
   			 \label{plot8_leftright}
		\end{figure}		
		\subsection{$y=f(x)=\frac{\ln x}{x^3}$}%9
		\rule{\textwidth}{0.2pt}
		\begin{center}\colorbox{slight}{
				\begin{tabular}{p{0.9\textwidth}}
					\MJHmarker{\textbf{MATLAB語法 :}}\\					
					f = @(x) log(x)*x.$\^$(-3);\\
					plot9 = fplot(f);\\
					line([-10 10] ,[0 0],'linestyle',':','color','red');\\
					grid\\
					title("f(x)=ln(x)/x$\backslash\^$3");\\
					set(plot9,'linewidth',2);\\
					xlim([-2 10]);\\
					ylim([-10 2]);\\
				\end{tabular}
			}
			\end{center}
			\newpage			
		函數圖形呈現如下:
		\begin{figure}[H]	
		 	 \centering	 			 	 
   			 \includegraphics[width=1\textwidth]{\imgdir homework2_OrginalPlot9.jpg} 
   			 \caption{$y=f(x)=\frac{\ln x}{x^3}$}   		
   			 \label{plot9}   			 		 
		\end{figure}
		
		此例一樣須注意向量中,除法不易執行,因此一樣使用$-3$次方完成除法的部分,而此圖看似與$x$軸			有漸進線,卻無法確定是否存在交點,因此刻意放大觀察,如下圖:
		\begin{figure}[H]	
		 	 \centering	 			 	 
   			 \includegraphics[width=1\textwidth]{\imgdir homework2_plot9Zoom.jpg} 
   			 \caption{$y=f(x)=\frac{\ln x}{x^3}$}   		
   			 \label{plot9zoom}   			 		 
		\end{figure}
		
		確定有與$x$軸有交點,並非漸進線,更進一步觀察:
		\begin{figure}[H]	
		 	 \centering	 			 	 
   			 \includegraphics[width=1\textwidth]{\imgdir homework2_plot9Zoom2.jpg} 
   			 \caption{$y=f(x)=\frac{\ln x}{x^3}$}   		
   			 \label{plot9zoom2}   			 		 
		\end{figure}
		
		約在$x=1$時存在交點。
		
		\subsection{$y=f(x)=3,1\leq x \leq5$}%10
		\rule{\textwidth}{0.2pt}
		\begin{center}\colorbox{slight}{
				\begin{tabular}{p{0.9\textwidth}}
					\MJHmarker{\textbf{MATLAB語法 :}}\\					
					line([1 5] ,[3 3],'color','red','linewidth',3);\\
					title("y=f(x)=3");\\
					xlim([0.5 5.5])\\
					grid\\
				\end{tabular}
			}
			\end{center}	
				
		函數圖形呈現如下:
		\begin{figure}[H]	
		 	 \centering	 			 	 
   			 \includegraphics[width=1\textwidth]{\imgdir homework2_plot10.jpg} 
   			 \caption{$y=f(x)=3,1\leq x \leq5$}   		
   			 \label{plot10}   			 		 
		\end{figure}
		此例較為單純,僅須注意$x$的定義域僅限於$1$至$5$,因此設定'line'時,$x$範圍以"[1 5]"表			現。
		
		\subsection{$x^2+y^2=1$}%11
		\rule{\textwidth}{0.2pt}
		\begin{center}\colorbox{slight}{
				\begin{tabular}{p{0.9\textwidth}}
					\MJHmarker{\textbf{MATLAB語法 :}}\\					
					f=@(x,y) x.$\^$2+y.$\^$2-1;\\
					plot11=fimplicit(f , [-1.5 1.5 -1.5 1.5]);\\
					set(plot11,'marker','s');\\
					grid\\
					title('x$\backslash\^$2+y$\backslash\^$2=1')\\
				\end{tabular}
			}
			\end{center}	
		\newpage	
		函數圖形呈現如下:
		\begin{figure}[H]	
		 	 \centering	 			 	 
   			 \includegraphics[width=1\textwidth]{\imgdir homework2_plot11.jpg} 
   			 \caption{$x^2+y^2=1$}   		
   			 \label{plot11}   			 		 
		\end{figure}
		
		本例子需要用到"fimplicit"輔佐",有別於先前僅是透過"fplot"可完成,由於此例是方程式,並非			多項式,因此需將等號左右整理成類似$ax+by+c=0$之形式,將左式帶入先前變數f,再藉					由"fimplicit"將此多項式形式轉成方程式存入變數"plot11",而其中後面"[-1.5 1.5 -1.5 				1.5]"則是設定$x,y$界限,亦可用"set"設定,最後作法和先前一致即可完成方程式圖形。
		\newpage
		\subsection{正方形}%12
		\rule{\textwidth}{0.2pt}
		\begin{center}\colorbox{slight}{
				\begin{tabular}{p{0.9\textwidth}}
					\MJHmarker{\textbf{MATLAB語法 :}}\\					
					title('正方形');\\
					grid;\\
					xlim([0 3])\\
					ylim([0 3])\\
					line([1 2] ,[1 1],'color','red','linewidth',3);\\
					line([1 2] ,[2 2],'color','red','linewidth',3);\\
					line([1 1] ,[1 2],'color','red','linewidth',3);\\
					line([2 2] ,[1 2],'color','red','linewidth',3);\\
				\end{tabular}
			}
			\end{center}	
				
		函數圖形呈現如下:
		\begin{figure}[H]	
		 	 \centering	 			 	 
   			 \includegraphics[width=1\textwidth]{\imgdir homework2_plot12.jpg} 
   			 \caption{正方形}   		
   			 \label{plot12}   			 		 
		\end{figure}
		
		先前多次以"line"做出水平,垂直線等,而正方形亦可透過水平與垂直線形成,僅需控制$x$與$y$的			範圍即可達成!
	\section{特殊函數圖形呈現}
		\subsection{$f(y)=\frac{1}{\beta}e^{\frac{-y}{\beta}},0\leq y \leq \infty$}%o1
		
		\begin{center}\colorbox{slight}{
				\begin{tabular}{p{0.9\textwidth}}
					\MJHmarker{\textbf{MATLAB語法 :}}\\					
					x=[0 :0.1:5];\\
					beta=1;\\
					y=(beta$\^$(-1))*exp((-x)*(beta$\^$(-1)));\\
					plotO1 = plot(x,y);\\
					set(plotO1,'linewidth',3);\\
					ylim([0 1])\\
					title("Exponential Probability Distribution ($\backslash$beta = 						1)")\\
					xlabel("x")\\
					ylabel('Probability Density Function')\\
					grid;\\
					set(gca,'fontsize',14);\\
				\end{tabular}
			}
			\end{center}
		函數圖形呈現如下:
		\begin{figure}[H]	
		 	 \centering	 			 	 
   			 \includegraphics[width=1\textwidth]{\imgdir h2plotO1.jpg} 
   			 \caption{$f(y)=\frac{1}{\beta}e^{\frac{-y}{\beta}},0\leq y \leq \infty$} 	
   			 \label{h2plotO1}   			 		 
		\end{figure}		
		\subsection{$f(y)=\left[ \frac{\gamma (\alpha + \beta)}{\gamma (\alpha)					\gamma (\beta)} \right] y^{\alpha -1}(1-y)^{\beta -1},0 \leq y \leq 1$}%o2
		
		\begin{center}\colorbox{slight}{
				\begin{tabular}{p{0.9\textwidth}}
					\MJHmarker{\textbf{MATLAB語法 :}}\\					
					alpha = 1;beta =2;x = [0:0.01:1];\\		
					temp=gamma((alpha+beta)/(gamma(alpha)*gamma(beta)));\\
					y1 = temp.*(x.$\^$(alpha-1)).*(1-x).$\^$(beta-1);\\
					alpha = 1;beta = 3;\\					
					y2 = temp.*(x.$\^$(alpha-1)).*(1-x).$\^$(beta-1);\\
					alpha = 2;beta = 2;\\					
					y3 = temp.*(x.$\^$(alpha-1)).*(1-x).$\^$(beta-1);\\
					alpha = 2;beta = 3;\\					
					y4 = temp.*(x.$\^$(alpha-1)).*(1-x).$\^$(beta-1);\\
					alpha = 3;beta = 1;\\					
					y5 = temp.*(x.$\^$(alpha-1)).*(1-x).$\^$(beta-1);\\
					alpha = 6;beta = 1;\\					
					y6 = temp.*(x.$\^$(alpha-1)).*(1-x).$\^$(beta-1);\\
					plot(x,y1,x,y2,x,y3,x,y4,x,y5,x,y6,'linewidth',2)\\
					xlim([0,1]);ylim([0,2]);grid\\	
					legend("$\backslash$ alpha =1 $\backslash$ beta = 2", "$\backslash						$alpha =1 $\backslash$beta = 3","$\backslash$alpha =2$\backslash						$beta = 2","$\backslash$alpha =2 $\backslash$beta = 3","$								\backslash$alpha =3 $\backslash$beta = 1","$\backslash$alpha =6 						$\backslash$beta = 1")\\
					title("Beta Dist.");xlabel("x")\\					
					ylabel('Probability Density Function')\\
					set(gca,'fontsize',14);\\
				\end{tabular}
			}
			\end{center}	
		
		函數圖形呈現如下:
		\begin{figure}[H]	
		 	 \centering	 			 	 
   			 \includegraphics[width=1\textwidth]{\imgdir h2plotO2.jpg} 
   			 \caption{Beta 分配} 	
   			 \label{h2plotO2}   			 		 
		\end{figure}		
		\subsection{$X^2$ 分配}%o3		
		\begin{center}\colorbox{slight}{
				\begin{tabular}{p{0.9\textwidth}}
					\MJHmarker{\textbf{MATLAB語法 :}}\\					
					v=5;alpha = v/2;beta =2;x = [0:0.1:20];\\
					temp=1/((gamma(alpha))*(beta$\^$alpha));\\
					y = temp.*(x.$\^$(alpha-1)).*exp(-x./beta);\\
					plot(x,y,'linewidth',3,'color','red');hold on;\\
					title("Chi-square Dist.");\\
					xlabel("x");grid;\\
					ylabel('Probability Density Function')\\
					set(gca,'fontsize',14);\\					
					myBar=bar(x,y);hold off;\\
					color \_ background=['c' 'm' 'y' 'k' 'r' 'g' 'b'];\\
					set(myBar,'FaceColor',color\_ background(7));\\
				\end{tabular}
			}
			\end{center}	
				
		函數圖形呈現如下:
		\begin{figure}[H]	
		 	 \centering	 			 	 
   			 \includegraphics[width=1\textwidth]{\imgdir h2plotO3.jpg} 
   			 \caption{$X^2$ 分配} 	
   			 \label{h2plotO3}   			 		 
		\end{figure}		
		其中利用了"bar" 讓函數底下面積也能夠一併顯示。而"colorbackground"則先將所有顏色存入後,			未來方便使用,直接從此array內一一叫出即可。
	\section{其他參數與圖表}
		\subsection{Marker 使用}%o4
		\rule{\textwidth}{0.2pt}
		\begin{center}\colorbox{slight}{
				\begin{tabular}{p{0.9\textwidth}}
					\MJHmarker{\textbf{MATLAB語法 :}}\\					
					f = @(x) x*sqrt(4-x.$\^$(2));\\
					plot8 = fplot(f);\\
					ylim([-2.5,2.5]);\\
					xlim([-3,3]);\\
					set(plot8,'linewidth',2);\\
					title("f(x)=x*sqrt(4-x$\backslash\^$2)");\\
					line([-10 10] ,[2 2],'linestyle',':','color','red');\\
					line([-10 10] ,[-2 -2],'linestyle',':','color','red');\\
					set(plot8, 'Marker', '<');\\
					set(plot8, 'MarkerSize', 18); \\
					set(plot8,'color','b');\\
					set(plot8,'MarkerEdgeColor','g');\\
					set(plot8,'MarkerFaceColor',[1 .6 .6]);\\
					grid\\
				\end{tabular}
			}
			\end{center}
		此例"Marker"設定為"<",也就是三角形的圖式,以下總共六種範例以供參考,皆式調整"Marker"參			數可能,其中包含"+","*","o","p"等等。
		\begin{figure}[H]
    		 \centering
      		 \subfloat[--gs]{
       		 \includegraphics[scale=0.15]{\imgdir h2plotO4_1.jpg}}
        	 \subfloat[-o]{
       		 \includegraphics[scale=0.15]{\imgdir h2plotO4_2.jpg}}
   			 \caption{Marker 展示 "--gs"和"--o"}
   			 \label{plot412}
		\end{figure}	
		\begin{figure}[H]
    		 \centering
      		 \subfloat[*]{
       		 \includegraphics[scale=0.15]{\imgdir h2plotO4_3.jpg}}
        	 \subfloat[h]{
       		 \includegraphics[scale=0.15]{\imgdir h2plotO4_4.jpg}}
   			 \caption{Marker 展示 "*"和"h"}
   			 \label{plot434}
		\end{figure}	
		\begin{figure}[H]
    		 \centering
      		 \subfloat[+]{
       		 \includegraphics[scale=0.15]{\imgdir h2plotO4_5.jpg}}
        	 \subfloat[<]{
       		 \includegraphics[scale=0.15]{\imgdir h2plotO4_6.jpg}}
   			 \caption{Marker 展示 "+"和"<"}
   			 \label{plot456}
		\end{figure}	
		\subsection{linestyle 使用}%o4
		\rule{\textwidth}{0.2pt}
		\begin{center}\colorbox{slight}{
				\begin{tabular}{p{0.9\textwidth}}
					\MJHmarker{\textbf{MATLAB語法 :}}\\					
					x = 0:pi/100:2*pi;\\
					y1 = sin(x);\\
					y2 = sin(x-0.25);\\
					y3 = sin(x-0.5);\\
					y4 = sin(x-0.75);\\
					y5 = sin(x-1);\\
					y6 = sin(x-1.25);\\
					figure\\
					plot(x,y1,x,y2,'--',x,y3,':',x,y4,'b--o',x,y5,'c*',x,y6,'-.')\\
					title('LineStyle Display')\\
					set(gca,'fontsize',16);\\
					grid\\
				\end{tabular}
			}
			\end{center}
		圖形呈現如下:
		\begin{figure}[H]	
		 	 \centering	 			 	 
   			 \includegraphics[width=1\textwidth]{\imgdir h2plotO5.jpg} 
   			 \caption{linestyle 展示} 	
   			 \label{h2plotO5}   			 		 
		\end{figure}
		透過"linestyle"可以設定不同種類的線,其中包含"--","c*","-."等等。
		\subsection{Bar 使用}%o4		
		\rule{\textwidth}{0.2pt}
		\begin{center}\colorbox{slight}{
				\begin{tabular}{p{0.9\textwidth}}
					\MJHmarker{\textbf{MATLAB語法 :}}\\					
					x = 1900:10:2000;\\
					y = [100 91 50 123.5 131 20 179 203 200 249 100];\\
					bar(x,y,'facecolor','black')\\
				\end{tabular}
			}
			\end{center}
		圖形呈現如下:
		\begin{figure}[H]	
		 	 \centering	 			 	 
   			 \includegraphics[width=1\textwidth]{\imgdir h2plotB1.jpg} 
   			 \caption{一般常見的Bar 展示 } 	
   			 \label{h2plotB1}   			 		 
		\end{figure}
		\begin{center}\colorbox{slight}{
				\begin{tabular}{p{0.9\textwidth}}
					\MJHmarker{\textbf{MATLAB語法 :}}\\					
					y = [2 2 3; 2 5 6; 2 8 9; 2 11 12];\\
					bar(y,'stacked')\\
				\end{tabular}
			}
			\end{center}
		圖形呈現如下:
		\begin{figure}[H]	
		 	 \centering	 			 	 
   			 \includegraphics[width=1\textwidth]{\imgdir h2plotB2.jpg} 
   			 \caption{多條Bar重疊 展示} 	
   			 \label{h2plotB2}   			 		 
		\end{figure}
		\rule{\textwidth}{0.2pt}
		\begin{center}\colorbox{slight}{
				\begin{tabular}{p{0.9\textwidth}}
					\MJHmarker{\textbf{MATLAB語法 :}}\\					
					x = [1 2 3];\\
					vals = [20 15 6; 11 23 26];\\
					b = bar(x,vals);\\
					set(b(1),'FaceColor','y');\\
					set(b(2),'FaceColor','g');\\
					grid;\\
					title("My Bar");\\
					set(gca,'fontsize',16);\\
				\end{tabular}
			}
			\end{center}
		圖形呈現如下:
		\begin{figure}[H]	
		 	 \centering	 			 	 
   			 \includegraphics[width=1\textwidth]{\imgdir h2plotB3.jpg} 
   			 \caption{並排Bar 展示} 	
   			 \label{h2plotB3}   			 		 
		\end{figure}
		\subsection{Histogram 使用}%o4
		\begin{center}\colorbox{slight}{
				\begin{tabular}{p{0.9\textwidth}}
					\MJHmarker{\textbf{MATLAB語法 :}}\\					
					x = randn(1000,5); \\
					nbins = 7;\\
					hist(x,nbins);\\
				\end{tabular}
			}
			\end{center}
		圖形呈現如下:
		\begin{figure}[H]	
		 	 \centering	 			 	 
   			 \includegraphics[width=1\textwidth]{\imgdir h2plotH1.jpg} 
   			 \caption{Histogram 展示} 	
   			 \label{h2plotH1}   			 		 
		\end{figure}
		\subsection{Scatter 使用}%o4		
		\rule{\textwidth}{0.2pt}
		\begin{center}\colorbox{slight}{
				\begin{tabular}{p{0.9\textwidth}}
					\MJHmarker{\textbf{MATLAB語法 :}}\\					
					x = linspace(0,3*pi,200);\\
					y = cos(x) + rand(1,200);\\
					sz = 50;\\
					c = linspace(1,10,length(x));\\
					scatter(x,y,sz,c,'filled')\\
				\end{tabular}
			}
			\end{center}
		圖形呈現如下:
		\begin{figure}[H]	
		 	 \centering	 			 	 
   			 \includegraphics[width=1\textwidth]{\imgdir h2plotS1.jpg} 
   			 \caption{Scatter 展示} 	
   			 \label{h2plotS1}   			 		 
		\end{figure}
		其中可以利用scatter中第三個參數調整點的大小,第四個參數調整顏色,第五個參數調整是否填滿。
		\subsection{Pie 使用}%o4		
		\rule{\textwidth}{0.2pt}
		\begin{center}\colorbox{slight}{
				\begin{tabular}{p{0.9\textwidth}}
					\MJHmarker{\textbf{MATLAB語法 :}}\\					
					X = [2 2 0.3 5 1];\\
					explode = [0 0 1 1 0];\\
					pie(X,explode)\\
				\end{tabular}
			}
			\end{center}	
			
		圖形呈現如下:
		\begin{figure}[H]	
		 	 \centering	 			 	 
   			 \includegraphics[width=1\textwidth]{\imgdir h2plotP1.jpg} 
   			 \caption{Pie分割 展示} 	
   			 \label{h2plotP1}   			 		 
		\end{figure}
		\rule{\textwidth}{0.2pt}
		\begin{center}\colorbox{slight}{
				\begin{tabular}{p{0.9\textwidth}}
					\MJHmarker{\textbf{MATLAB語法 :}}\\					
					X = [2 2 0.3 5 1];\\
					pie(X)\\
				\end{tabular}
			}
			\end{center}	
			
		圖形呈現如下:
		\begin{figure}[H]	
		 	 \centering	 			 	 
   			 \includegraphics[width=1\textwidth]{\imgdir h2plotP2.jpg} 
   			 \caption{一般常見的Pie 展示} 	
   			 \label{h2plotP2}   			 		 
		\end{figure}
		\rule{\textwidth}{0.2pt}
		\begin{center}\colorbox{slight}{
				\begin{tabular}{p{0.9\textwidth}}
					\MJHmarker{\textbf{MATLAB語法 :}}\\					
					X = [0.3 0.4 0.1];\\
					pie(X)\\
				\end{tabular}
			}
			\end{center}	
			
		圖形呈現如下:
		\begin{figure}[H]	
		 	 \centering	 			 	 
   			 \includegraphics[width=1\textwidth]{\imgdir h2plotP3.jpg} 
   			 \caption{特定缺口之Pie 展示} 	
   			 \label{h2plotP3}   			 		 
		\end{figure}
		
	\section{結論 {\ESITC{Conclusion}}}
		透過MATLAB可以實現各種不同數學函數的圖表產生,並且有別於其他語言,MATLAB更可以透過產生後			的圖表,自行利用內建圖形化界面設定圖內參數等等,不需要一一透過指令即可達成,而指令上來說,				與大部分程式大同小異,若有程式基礎,大概可以猜出大部分內容,並且此軟體也支援許多深度學				習,統計,機器學習方面的APP,若需要亦可從介面中下載,可說是非常方便,在使用方便度與實用性來			說不亞於python或R,而此次也展現其在繪圖上功能的完善之處,與其多樣性,可說是對數學或函數圖			形上最友善的程式語言之一。		
%\end{document}





