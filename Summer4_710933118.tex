%\input{../Jiang_Preamble}
%\title{\MJHmarker 監督式學習 \\ \Large{\color{darkblue}{迴歸分類}}}
%\author{{\MJH 江柏學}}
%\date{\today}  
%\begin{document}
%	\maketitle
%	\fontsize{12}{22pt}\selectfont
\chapter{ \MJH 監督式學習之迴歸分類}
	\textbf{監督式學習 (Supervised learning)} 是電腦從標籤化 (labeled) 的資訊中分析模式後做出		預測的學習方式。標記過的資料就好比標準答案,而本文所探討的方式,正屬於監督式學習。假設我們擁有一		組資料集,我們想知道\textbf{若是存在新的一筆資料,它是否也屬於此資料集},或是\textbf{對此資料		集進行分類},有助於\textbf{判斷新的資料歸屬的類別},舉例來說,若我們能得到大量醫院資料,並找出		幾項誘發高血壓的因素($x$),是否我們就能歸類有何種$x$傾向的人,較可能罹患高血壓,而面對此問題,		最重要的一點,就是找出合適的分類器,而這也是本文所要探討的,\textbf{從統計的觀點建立合適的分類		器}。
	\section{迴歸模型}
		回到上述的問題中,假設我們的資料集中,有一群屬性資料($x$)以及最後目標變數($y$),要如何得			知擁有何種$x$會形成何種$y$,而在面對此問題時,我們先簡單將資料呈現於平面圖形上:
		\begin{figure}[H]	
			\centering	 			 	 
   			\includegraphics[scale=0.26]{\imgdir sample_ori.jpg} 
   			\caption{範例資料集}
   			\label{fig:sample_ori}   			 		 
		\end{figure}
		圖\ref{fig:sample_ori}\ 我們假設$x$軸為變數$x1$,$y$軸為變數$x2$,而在此我們以顏			色還有圖形簡單區分$y$變向,可以明顯的看出,圖形呈現兩種不同的類別,並彷彿能以一條線區分開此			二群,但要以何種線段區分呢,本文將一一介紹。		
		\subsection{簡單線性迴歸}
			我們首先假設資料集能以$y=\beta_0+ \beta_1 x_1+\beta_2 x_2,\forall y \in 					\{0,1\}$配適,而在此模型下,當$y<0.5$我們就區分為0群,當$y>0.5$就區分成1群,如此便				可得一分類模型,而在此模型底下我們僅需估計出$\beta_0$、$\beta_1$、$\beta_2$即可,				並且令$y=0.5$時,便是此分類器區分之依據,也是圖\ref{fig:sample_ori}中,我們所需的				分類線段,而在此我們利用MATLAB實作幫助研究,透過模擬資料集,一步一步探討此問題:\\		
			\begin{enumerate}[Step 1:]
				\item {\textbf{取得並了解資料集}\\
					在此我們透過MATLAB取得既有的資料集,並對此資料集先進行簡單的觀察,看有無問						題,以及觀察變數大概落的位置,或是先觀察哪些是類別資料哪些是連續資料,如圖							\ref{fig:step1}
					\begin{figure}[H]	
		 		 		\centering	 			 	 
   				 		\includegraphics[width=1\textwidth]{\imgdir step1_box.jpg} 
   			 			\caption{REG範例資料大致分佈}   		
   			 			\label{fig:step1}   			 		 
					\end{figure}
					由圖\ref{fig:step1}可看出兩變數($x_1,x_2$)平均數差不多,但$x_1$變異程度						較大,且可能為右偏分布,而$x2$則看似對稱,變異程度也較小,兩變數皆無離群值,						看起來無較不妥之處。
				}
				\item {\textbf{定義Design matrix以及估計Coefficien vector}\\
					在此我們先定義\textbf{Design matrix}也就是矩陣$X$,令先前資料集中$x_1$及						$x_2$為此設計矩陣之兩變數後,矩陣型態如圖\ref{fig:step2}:
					\begin{figure}[H]	
		 		 		\centering	 			 	 
   				 		\includegraphics[width=0.7\textwidth]{\imgdir step2_xMat.jpg} 
   			 			\caption{Design Matrix}   		
   			 			\label{fig:step2}   			 		 
					\end{figure}
					接著有了設計矩陣後,我們便可透過統計的觀念估計$\beta$,如式\ref{eq:beta}
					\begin{equation}\label{eq:beta}
 						\hat{\beta}=(x^T x)^{-1}x^T y
 					\end{equation}
 					如此就可以取得完整的線性模型方程式,而每一點也能夠計算出其\textbf{配適值							(fitted value)},亦可計算出殘差等等資訊。
				}
				\item {\textbf{建立模型與分類器}\\
					最後,由上步的估計值,便可建立出完整迴歸模型,如式\ref{eq:reg}
					\begin{equation}\label{eq:reg}
 						\hat{y}=0.7419-0.1477x_1+0.0835x_2
 					\end{equation}
 					而有了模型後,我們透過$y=0.5$建立出分類器,而此分類器即是方程式									\ref{eq:model}
 					\begin{equation}\label{eq:model} 						
 						x_2=-2.897+1.769x_1
 					\end{equation}
 					最後,便可繪製一條分類線,而此線依據$y$值將資料集區分成兩區塊,如圖								\ref{fig:model1}:
 					\begin{figure}[H]	
		 		 		\centering	 			 	 
   				 		\includegraphics[width=1\textwidth]{\imgdir model1.jpg} 
   			 			\caption{分類器}   		
   			 			\label{fig:model1}   			 		 
					\end{figure}
					如此我們便透過迴歸的概念完成一條簡易的分類器,將原始資料區分成兩類,並且建立						模型,因此倘若有新資料,我們亦能推算出新資料的類別,而其中所有程式碼如下所							示:
					\begin{center}\colorbox{slight}{
						\begin{tabular}{p{0.9\textwidth}}
							\MJHmarker{\textbf{\color{darkblue}{MATLAB語法 :}}}\\		
							D = load('la\_ 1.txt');\\							
							x1=D(:,1); x2=D(:,2); y=D(:,3);\\
							gscatter(x1,x2,y,'br','ox',10);\\
							N=size(D,1);x=[ones(N,1),x1,x2];\\							
							b=(x'*x)$\backslash$(x'*y)\\
							f=@(x) (-b(2)/b(3))*x+(0.5-b(1))/b(3);\\
							hold on;\\
							fplot(f,[0 6],'LineWidth',3);\\
							hold off;grid;\\							
							title('分類器');legend('hide');\\							
							set(gca,'fontsize',20);\\
						\end{tabular}
					}
					\end{center}
				}
			\end{enumerate}
			我們完成迴歸模型之分類器後,更想知道此分類器判斷的到底對不對,若是有錯誤存在,那麼錯誤				率大概是多少,因此我們實際透過資料集中的$y$和我們所得到的配適值$\hat{y}$相減,並取絕				對值加總後,即得到錯誤的筆數,其中我們令$\hat{y}<0.5$為$0$,$\hat{y}>0.5$為$1$,				以此資料集為例,錯誤筆數有即有12筆,而總筆數為200筆,因此錯誤率為$0.06$,而我們也標明				錯誤判斷之資料如圖\ref{fig:errorData}:
			\begin{figure}[H]	
		 		 \centering	 			 	 
   				 \includegraphics[width=1\textwidth]{\imgdir errorData.jpg} 
   			 	\caption{錯誤資料}   		
   			 	\label{fig:errorData}   			 		 
			\end{figure}
			圖\ref{fig:errorData}\ 中,以綠色點標示的資料,即為錯誤判斷的資料,而以下為建					模後並取得配適值($\hat{y}$)後的程式碼:
			\begin{center}\colorbox{slight}{
				\begin{tabular}{p{0.9\textwidth}}
					\MJHmarker{\textbf{\color{darkblue}{MATLAB語法 :}}}\\		
					temp=[x,y,yHat];\\
					errorData=temp(temp(:,3)\ \~\  =temp(:,4),1:2)\\
					gscatter(x(:,1),x(:,2),y,'br',"ox",10);\\
					f=@(x) (-b(2)/b(3))*x+(0.5-b(1))/b(3);\\
					hold on;\\
					fplot(f,[0 6],'LineWidth',3)\\
					plot(errorData(:,1),errorData(:,2),'+','Color','g',"LineWidth",5);
					\\
					hold off;grid;legend('hide');\\				
				\end{tabular}
			}
			\end{center}
		\subsection{加廣型迴歸}
		然而,在日常生活中,並非所有資料都能用簡單線性模型套用,而線性模型也不一定是最好的選項,因			此衍生出其他模型種類,而在此也將介紹其中一種,\textbf{加廣型迴歸},有別於線性模型,加廣型			有更豐富的可能性,產生的圖形也並非直線,而在繪圖之前,我們先了解其在迴歸上的內涵。
		\\
		\\
		由於加廣型的概念是認為最初假設的函數,可能包含二次項,可能具有共線性,因此我們將最初的假設			加以修改,如式\ref{eq:augMdl}:
		\begin{equation}\label{eq:augMdl} 
		y=\beta_0+ \beta_1 x_1+\beta_2 x_2+\beta_3 x_1 x_2+\beta_4 x_1^2+\beta_5 				x_2^2
		\end{equation}
		式\ref{eq:augMdl}\ 除了基本兩變數之外,更加入其平方項以及相乘項,而此方式效果呈現如何,			將以MATLAB作展示,和簡單線型迴歸的做法相同,我們透過既有的資料集,進行程式實作:
		\bigskip
		\begin{enumerate}[Step 1:]			
			\item {\textbf{定義Design matrix}\\
				首先我們已經確定了資料集如上節所示,但由於考量變數由$2$項增至$5$項,因此我們的設					計矩陣同時也更正,如圖\ref{fig:augXMat}:
				\begin{figure}[H]	
		 		 	\centering	 			 	 
   				 	\includegraphics[width=1\textwidth]{\imgdir augXMat.jpg} 
   			 		\caption{Augmented Model Design Matrix(前十筆)}   		
   			 		\label{fig:augXMat}   			 		 
				\end{figure}
				圖\ref{fig:augXMat}\ 中明顯相較於圖\ref{fig:step2}\ 增加許多變項,因此迴歸模					型也	會較為複雜。
			}
			\item {\textbf{估計Coefficien vector}\\
				在此和簡單線性迴歸一樣,需估計$\beta$,而慶幸的是,此步驟在程式執行上和先前相同,					可以參考式\ref{eq:beta},而在此以MATLAB估計出的參數如表\ref{table:beta}:
				\begin{table}[h]				
					\caption{$\hat{\beta}$ vector}\label{table:beta}
					\centering
					\extrarowheight=8pt
					\begin{tabular}{llllll} 					
					\hline
					$\beta_0$\ &$\beta_1$ \ &$\beta_2$ \ &$\beta_3$ \ &$\beta_4$ \ &$						\beta_5$ \\ \hline	
					0.8051\ & -0.2287\ & 0.0943\ & -0.0043\ & 0.0125\ &-0.0016 \\
					\hline					
					\end{tabular}
				\end{table}
			}
			\item {\textbf{建立模型與分類器}\\
				在我們估計出參數後,建立模型如式\ref{eq:augReg}\ 所示:
				\begin{equation}\label{eq:augReg}
					\hat{y} = 0.8051-0.2287x_1+0.0943x_2-0.0043x_1 											x_2+0.0125x_1^2-0.0016x_2^2
				\end{equation}
				同理,一樣透過$y=0.5$來建立分類器,並且利用MATLAB繪製分類線,如圖								\ref{fig:augMdl}:
				\begin{figure}[H]	
		 		 	\centering	 			 	 
   				 	\includegraphics[width=1\textwidth]{\imgdir augMdl.jpg} 
   			 		\caption{Augmented Model}   		
   			 		\label{fig:augMdl}   			 		 
				\end{figure}
				而此例之曲線弧度雖不明顯,但仍可看出與圖\ref{fig:model1}之間的不同。			
			}
		\end{enumerate}
		同樣地,我們一樣關心著此分類器的錯誤率高低,成效如何,因此透過MATLAB實作得出此模型錯誤率依			然為$0.06$,因此我們得知在此資料集當中,簡單線性迴歸的分類器與加廣型迴歸之分類器,錯誤率一			樣,成效相去不遠,然而簡單線性迴歸在實作上較容易,效率較高,因此最此資料集而言,透過簡單線			性迴歸的方式可能更好。
	\section{模擬資料}
		我們剛才所用的資料集都是既有的資料集,然而,在日常生活中,以有限的金錢與時間考量下,我們可			能無法取得大量資料集供分析實驗,因此必須考量到如何透過MATLAB程式,自行模擬出資料集,而再透			過資料集進行分析與實驗。
		\\		
		在MATLAB中,透過\textbf{mvnrnd}指令可以幫助我們產生多維的常態分佈資料,而我們以下也將利			用此指令幫助我們模擬新的資料集,並且進一步討論此資料集對於研究的優劣程度,指令如下所示:
		\begin{center}\colorbox{slight}{
			\begin{tabular}{p{0.9\textwidth}}
				\MJHmarker{\textbf{\color{darkblue}{MATLAB語法 :}}}\\		
					n1=200;n2=200;mu1=[0 1];mu2=[4 6];\\					
					S1=[1 0;0 1];S2=[1 0;0 1];y=[zeros(n1,1);ones(n2,1)];\\				
					A=mvnrnd(mu1,S1,n1);B=mvnrnd(mu2,S2,n2);\\
					X=[A;B]\\
					gscatter(X(:,1),X(:,2),y,'br','ox')\\

			\end{tabular}
		}
		\end{center}
		此程式先定義樣本數n1與n1,接著定義常態分配之參數$\mu$與$\sigma$,還有所要分的類別y,最後			透過\textbf{mvnrnd}以及參數,產生多維常態分佈的資料,最後即可利用此資料集繪圖,如圖				\ref{fig:badData}:
		\begin{figure}[H]	
		 	\centering	 			 	 
   			\includegraphics[width=1\textwidth]{\imgdir badData.jpg} 
   			\caption{模擬資料之散佈圖}   		
   			\label{fig:badData}   			 		 
		\end{figure}
		然而,這樣的資料對於本文所探討之分類器優劣並無幫助,由於兩群能明顯區分開來,無論何種分類				器,最終所得出的錯誤率都極低甚至為零,鑑別度不高,因此我們利用同樣的方式,反覆實作出以下幾			種不同的資料集,如圖\ref{fig:Data}\ 所示:
		\begin{figure}[H]
    		 	\centering
      			 \subfloat[]{
       			 \includegraphics[scale=0.15]{\imgdir data_2_3.jpg}}
        		 \subfloat[]{
       			 \includegraphics[scale=0.15]{\imgdir data_3_3.jpg}}
       			 \\
       			 \subfloat[]{
       			 \includegraphics[scale=0.15]{\imgdir data_3_5_08.jpg}}
       			 \subfloat[]{
       			 \includegraphics[scale=0.15]{\imgdir data_3_5_15.jpg}}
       			 \caption{模擬不同資料集}   
   				 \label{fig:Data}
		\end{figure}
		圖\ref{fig:Data}\ (a)中,資料太密集,較不易區分,而圖\ref{fig:Data}\ (b)中,兩群分太			開,分類器鑑別度也較低,因此以散佈圖直觀判斷,若要做分類器優劣比較,圖\ref{fig:Data}\ 			(c)與(d)可能會是較優先的選項,而以下也將示範如何透過自己模擬的資料集,比較上節討論的分類			器:
		\begin{itemize}	
			\item{\textbf{以圖\ref{fig:Data}\ (c)資料集為例:}
				我們同樣透過程式碼先觀察資料集大致分布,如圖\ref{fig:c1box}:
				\begin{figure}[H]	
		 			\centering	 			 	 
   					\includegraphics[scale=0.18]{\imgdir c1box.jpg} 
   					\caption{模擬資料大致分布}   		
   					\label{fig:c1box}   			 		 
				\end{figure}
				初步分析資料,並無離群值,也沒有較特別的異狀,且變數$x_2$較$x_1$變異程度高,而無					異狀後我們直接進行模型配適,而在此不同於先前建模方式,我們以MATLAB本身語法進行,					如下所示:
				\begin{center}\colorbox{slight}{
					\begin{tabular}{p{0.9\textwidth}}
						\MJHmarker{\textbf{\color{darkblue}{MATLAB語法 :}}}\\		
						set(gca,'fontsize',30);\\
						D = load('myData2.mat');x=D.X;y=D.y;\\
						boxplot([x(:,1),x(:,2)]);\\
						figure,hold on;\\
						gscatter(x(:,1),x(:,2),y,'br',"ox",10);\\
						mdl=fitlm(x,y);\\
						b = mdl.Coefficients.Estimate;\\
						f=@(x) (-b(2)/b(3))*x+(0.5-b(1))/b(3);\\
						fplot(f,'LineWidth',3);\\
						hold off;legend('hide');grid;\\	
					\end{tabular}
				}
				\end{center}
				以指令\textbf{fitlm}即可取得模型,再透過MATLAB整理好的物件,將$\beta$也取得,					最後顯示圖形如圖\ref{fig:c1mdl}:
				\begin{figure}[H]	
		 			\centering	 			 	 
   					\includegraphics[width=1\textwidth]{\imgdir c1mdl.jpg} 
   					\caption{模擬資料之簡單迴歸分類器}   		
   					\label{fig:c1mdl}   			 		 
				\end{figure}
				而為了進行比較,我們計算其錯誤率為0.0475,在400筆資料中,有19筆判斷錯誤,而接著					我們以加廣型分類器進行建模,程式如下:
				\begin{center}\colorbox{slight}{
					\begin{tabular}{p{0.9\textwidth}}
						\MJHmarker{\textbf{\color{darkblue}{MATLAB語法 :}}}\\		
						set(gca,'fontsize',30);\\
						D = load('myData2.mat');\\
						x=D.X;y=D.y;n=size(x,1);\\
						figure,hold on;\\
						gscatter(x(:,1),x(:,2),y,'br',"ox",10);\\
						mdl=fitlm(x,y,'quadratic');\\
						Ab = mdl.Coefficients.Estimate;\\
						g=@(x1,x2) Ab(1)-0.5+Ab(2)*x1+Ab(3)*x2+Ab(4)*...\\
 					  	x1.*x2+Ab(5)*x1.\^2+Ab(6)*x2.\^2;\\
						fimplicit(g,'LineWidth',3,'LineStyle',"--");\\
						hold off;legend('hide');grid;\\	
					\end{tabular}
				}
				\end{center}
				而在此例,加廣型明顯配適出一條分類曲線,如圖\ref{fig:c1augmdl}:
				\begin{figure}[H]	
		 			\centering	 			 	 
   					\includegraphics[width=1\textwidth]{\imgdir c1augmdl.jpg} 
   					\caption{模擬資料之加廣型迴歸分類器}   		
   					\label{fig:c1augmdl}   			 		 
				\end{figure}	
				而其錯誤筆數為18,錯誤率0.045,因此我們可以得知,以訓練資料集的正確率來說,加廣型					迴歸	的表現比較好,但兩者之間判斷錯誤僅差一筆,因此效率可能仍要討論,而圖							\ref{fig:c1twomdl}\ 彙整兩種圖形,可看出兩分類器在圖形表現上的差異。
				\begin{figure}[H]	
		 			\centering	 			 	 
   					\includegraphics[width=1\textwidth]{\imgdir c1twomdl.jpg} 
   					\caption{兩分類器表現差異}   		
   					\label{fig:c1twomdl}   			 		 
				\end{figure}
			}
			\item{\textbf{以圖\ref{fig:Data}\ (d)資料集為例:}
				在圖\ref{fig:Data}(d)資料集中,能明先觀察出兩群散佈程度有明顯的差異,而透過盒形					圖觀察資料如圖\ref{fig:c2box}\ 可見$x_2$散佈明顯較廣,且有離群值存在,但離群值					不多,因此資料來源可能仍為正常:
				\begin{figure}[H]	
		 			\centering	 			 	 
   					\includegraphics[width=1\textwidth]{\imgdir c2box.jpg} 
   					\caption{圖\ref{fig:Data}\ (d)資料散佈情形}   		
   					\label{fig:c2box}   			 		 
				\end{figure}				
				接著透過模型配適,並利用簡單迴歸分類,觀察分類器的成效,如圖\ref{fig:c2mdl}:
				\begin{figure}[H]	
		 			\centering	 			 	 
   					\includegraphics[width=1\textwidth]{\imgdir c2mdl.jpg} 
   					\caption{圖\ref{fig:Data}\ (d)資料集之簡單迴歸分類器}   		
   					\label{fig:c2mdl}   			 		 
				\end{figure}
				而由程式運算結果,其在400筆資料中,有14筆錯誤資料,其中錯誤率為0.035,而程式碼如					下所示:
				\begin{center}\colorbox{slight}{
					\begin{tabular}{p{0.9\textwidth}}
						\MJHmarker{\textbf{\color{darkblue}{MATLAB語法 :}}}\\		
						set(gca,'fontsize',30);\\
						D = load('myData1.mat');\\
						x=D.X;y=D.y;\\						
						n=size(x,1);\\
						boxplot([x(:,1),x(:,2)]);\\
						figure,hold on;\\
						gscatter(x(:,1),x(:,2),y,'br',"ox",10);\\
						mdl=fitlm(x,y);\\
						b = mdl.Coefficients.Estimate;\\
						f=@(x) (-b(2)/b(3))*x+(0.5-b(1))/b(3);\\
						fplot(f,'LineWidth',3);\\
						hold off;legend('hide');grid;\\
						yHat = mdl.Fitted;\\
						yHat(yHat<0.5)=0;yHat(yHat>0.5)=1;\\						
						errorMdl = sum(abs(y-yHat))\\
						errorMdl = sum(abs(y-yHat))/n\\
					\end{tabular}
				}
				\end{center}
				接著利用加廣型迴歸模型,實驗結果,錯誤筆數為12筆,錯誤率為0.03,而函數圖形如圖						\ref{fig:c2aug}
				\begin{figure}[H]	
		 			\centering	 			 	 
   					\includegraphics[width=1\textwidth]{\imgdir c2aug.jpg} 
   					\caption{圖\ref{fig:Data}\ (d)資料集之加廣型迴歸分類器}   		
   					\label{fig:c2aug}   			 		 
				\end{figure}
				而在此例中,以訓練資料集的正確率來說,加廣型較簡單迴歸優異,其錯誤筆數少兩筆,因此					在此例中,以加廣型迴歸做分類可能較優異。
			}		
		\end{itemize}
		最後,我們討論當資料超過兩群時,要如何以上述的迴歸線進行分類,我們也以簡單線性迴歸實作,如			圖\ref{fig:threemdl}:
		\begin{figure}[H]	
		 	\centering	 			 	 
   			\includegraphics[width=1\textwidth]{\imgdir threemdl.jpg} 
   			\caption{三群簡單迴歸分類器}   		
   			\label{fig:threemdl}   			 		 
		\end{figure}
		其中,我們是以三次兩群分類繪圖,因此在程式中有三個模型,以便繪出三種不同的分割線,如下:
		\begin{center}\colorbox{slight}{
					\begin{tabular}{p{0.9\textwidth}}
						\MJHmarker{\textbf{\color{darkblue}{MATLAB語法 :}}}\\		
						D=load("myData3.mat");\\
						x=D.X;y=D.y;\\						
						figure,hold on;\\
						gscatter(x(:,1),x(:,2),y,'rbg','xo \textless ',10)\\
\\
						mdl=fitlm(x(1:300,:),y(1:300,:));\\
						b = mdl.Coefficients.Estimate;\\
						f=@(x) (-b(2)/b(3))*x+(0.5-b(1))/b3);\\
						fplot(f,[1.15 3],'LineWidth',3);\\
\\
						mdl2=fitlm(x(101:end,:),y(101:end,:));\\
						b = mdl2.Coefficients.Estimate;\\
						f=@(x) (-b(2)/b(3))*x+(1.5-b(1))/b(3);\\
						fplot(f,[1.25 5],'LineWidth',3);\\
\\
						mdl3=fitlm([x(1:100,:);x(301:end,:)],														[y(1:100,:);y(301:end,:)]);\\
						b = mdl3.Coefficients.Estimate;\\
						f=@(x) (-b(2)/b(3))*x+(1-b(1))/b(3);\\
						fplot(f,[-3 1.65],'LineWidth',3);\\
\\
						hold off;legend('hide');\\						
						ylim([-5 12])grid\\						
					\end{tabular}
				}
		\end{center}
		其中,仔細觀察圖\ref{fig:threemdl},圖中有部分區域屬於三條線之外,而若是新的資					料於此,可能透過此分類器無法有效判斷,因此我們嘗試以加廣型分類器,來實測是否有機會					解決此問題,而實測結果如圖\ref{fig:threeaug}:
		\begin{figure}[H]	
		 	\centering	 			 	 
   			\includegraphics[width=1\textwidth]{\imgdir threeaug.jpg} 
   			\caption{三群加廣型分類器}   		
   			\label{fig:threeaug}   			 		 
		\end{figure}
		圖\ref{fig:threeaug}\ 大致和圖\ref{fig:threemdl}\ 差不多,也有三條分類線無法判斷之			處,因此可能仍需另找其他有效方式,解決此問題。
	\section{結論}
		在簡單線性迴歸分類器,以及加廣型分類器上,兩種方法對於解決兩群以上的分類問題時,成效都不盡			理想,但對於兩群來說,此二分類器的表現,無論是理論或是圖形呈現,都能達到高準確度,而同時也			能藉由分類來預測新資料的歸屬,其中最重要的除了程式設計上的邏輯能力,以及語法熟悉度外,更需			要掌握其中迴歸方程式的理論,以及參數的估計,兼具理論以及實作能力,展現統計與監督式學習,正			是本文所想探討的。
%\end{document}









