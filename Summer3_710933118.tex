%\input{../Jiang_Preamble}
%\title{\MJHmarker MATLAB實作 \\ \Large{\color{darkblue}{統計分配}}}
%\author{{\MJH 江柏學}}
%\date{\today}  
%\newtheorem{ex}{{\Cam Example}}
%\begin{document}
%	\maketitle
%	\fontsize{12}{22pt}\selectfont
\chapter{ \MJH MATLAB實作之統計分配}
	MATLAB在處理各種數學函數上,有不盡其數的語法能使用,甚至在繪圖上,皆不亞於其他程式語言,而今日		也在此以MATLAB為主要示範語法,來著手其在\textbf{\color{darkblue}{函數展現}},					\textbf{\color{darkblue}{繪圖討論}},\textbf{\color{darkblue}{亂數產生}}等等方面之應			用,除了討論基本語法的變化,繪圖美觀的琢磨,更討論函數圖形在統計上的意義,介紹在不同參數下,各個		分配變化多端的樣貌!\\
	\bigskip
	\section{分配函數}
		在統計學上,充滿各種不同類型的分配,每個分配都有各自獨樹一幟的絕美風貌,更是各自帶著專屬於			自己的參數,而每個參數的組成,又能形成各種各樣的不同分配形狀,這也是統計的迷人所在,而以下			也將討論各種不同的分配,如何利用MATLAB呈現,而不同的參數,又是如何形成新的圖形樣貌:
		\subsection{連續型函數}
			本文所討論的,又可稱作連續型隨機變數,在統計上,連續的分配涵蓋各式各樣層面,有在生活中				常見的\textbf{\color{darkblue}{常態分配}},檢定中佔據重要地位的								\textbf{\color{darkblue}{卡方分配}},亦或是變化多端的\textbf{\color{darkblue}				{貝塔分配}},都屬於連續型的一種,而以下也配合幾種範例,展示MATLAB在統計函數上的使用,				以及各式分配的樣貌:
			\begin{enumerate}			
				\item{\textbf{\MJHmarker{常態分配:}}\\
					常態分布有兩個參數,分別為\textbf{\color{darkblue}{位置參數}									\emph{(location)}}\quad $\mu$、\textbf{\color{darkblue}{尺度參數}							\emph{(scale)}}\quad $\sigma$。當$\mu = 0$且$\sigma = 1$時,就稱為標						準常態分布。

					令$X$為一連續隨機變數,若$X$符合常態分布,其機率密度分布函數(P.D.F)為:
					$$f(x)=\frac{1}{\sigma\sqrt{2\pi}}e^{-\frac{(x-\mu)^2}									{2\sigma^2}}, \;\; -\infty < x < \infty $$	
					\\
					常態分布的機率密度函數圖形:
					\begin{figure}[H]	
		 		 		\centering	 			 	 
   				 		\includegraphics[width=1\textwidth]{\imgdir normal01.jpg} 
   			 			\caption{常態分配$\mu = 0 , \sigma = 1$}   		
   			 			\label{normal01}   			 		 
					\end{figure}
					\bigskip
					圖\ref{normal01} 是當$\mu = 0$且$\sigma = 1$時的常態分佈圖形,可看出其						鐘型分布的樣子,且至高點即是平均數$\mu$,而在3個標準差中,包含了絕大部分的資						料,而如此函數呈現,在MATLAB上僅需短短幾行程式碼:
					\begin{center}\colorbox{slight}{
						\begin{tabular}{p{0.9\textwidth}}
							\MJHmarker{\textbf{\color{darkblue}{MATLAB語法 :}}}\\		
							f=@(x) normpdf(x,0,1);\\
							fplot(f,'LineWidth',3);\\
							grid;\\
							legend('$\backslash$mu = 0 , $\backslash$sigma = 1');\\
							title("Normal Dist.");\\
							xlabel("x");\\
							ylabel("f(x)");\\
							ylim([0 0.5]);\\
							set(gca,'fontsize',16);	\\			
						\end{tabular}
					}
					\end{center}
					\bigskip				
					除此之外,我們熟知變異數越大時,常態分配圖形呈現越為矮胖,而當變異數越小時,						常態分配則是越為高瘦,而此參數變化,我們也能以MATLAB呈現,如圖									\ref{normal_multi}:
					\begin{figure}[H]	
		 		 		\centering	 			 	 
   				 		\includegraphics[width=1\textwidth]{\imgdir normal_multi.jpg} 
   			 			\caption{多種常態分配}   		
   			 			\label{normal_multi}   			 		 
					\end{figure}
					\bigskip
					在MATLAB應用上,加入\textbf{\color{darkblue}{for迴圈}}的使用,讓多種不						同參數的程式碼,更加簡潔:
					\begin{center}\colorbox{slight}{
						\begin{tabular}{p{0.9\textwidth}}
							\MJHmarker{\textbf{\color{darkblue}{MATLAB語法 :}}}\\		
							figure,hold on;\\
							for i = 1:5\\
    							\quad f=@(x) normpdf(x,0,i);\\
   							 	\quad fplot(f,[-15 15],'LineWidth',3);\\
							end\\
							legend('$\backslash$sigma = 1','$\backslash$sigma = 2','$								\backslash$sigma = 3','$\backslash$sigma =4','$\backslash								$sigma = 5');\\
							title("Normal Dist.");\\
							xlabel("x");\\
							ylabel("f(x)");\\
							ylim([0 0.5]);\\
							set(gca,'fontsize',20);\\
							grid;\\
							hold off;	\\	
						\end{tabular}
					}
					\end{center}
				}	
				\rule{\textwidth}{0.2pt}
				\item{\textbf{\MJHmarker{T分配:}}\\
					在機率論和統計學中,\textbf{\color{darkblue}{司徒頓t-分布}									\emph{(Student's t-distribution)}}可簡稱為\textbf{\color{darkblue}						{t分	布}},用於根據小樣本來估計呈常態分布且變異數未知的總體的平均值。如果總體						變異數已知(例如在樣本數量足夠多時),則應該用常態分布來估計總體均值。
					\\
					\\
					在上述例子中提及,常態分配在$\mu = 0$,$\sigma=1$的情況下,又可稱作							\textbf{\color{darkblue}{標準常態分配}},而當T分配中,										\textbf{\color{darkblue}{自由度}}在$30$以上時,圖形呈現非常趨近標準常							態分配,如此函數關係,我們可以簡單的從MATLAB中觀察出來,但在這之前,我們得先						對於T分配的函數圖形有所了解,並且熟悉參數(自由度)改變時,T分配會如何變化,而						圖\ref{t}是當自由度由0.1至1時的變化:
					\begin{figure}[H]	
		 		 		\centering	 			 	 
   				 		\includegraphics[width=1\textwidth]{\imgdir t.jpg} 
   			 			\caption{T分配(自由度由$0.1$至$1$)}   		
   			 			\label{t}   			 		 
					\end{figure}
					而我們一樣由迴圈可以簡單完成不同參數的T分配,其中此分配的										\textbf{\color{darkblue}{\emph{(KEY WORD)}}}為\textbf{tpdf},而其						在MATLAB上程式碼如下:
					\begin{center}\colorbox{slight}{
						\begin{tabular}{p{0.9\textwidth}}
							\MJHmarker{\textbf{\color{darkblue}{MATLAB語法 :}}}\\		
							figure, hold on;grid;\\    						
   							title("自由度(v)0.1至1");sInterval=[-5 5];\\  
   							nu=[0.1:0.2:1]; n=length(nu);\\						    
						    for i=1:n\\
						        \quad f=@(x) tpdf(x,nu(i));  \\       
						        \quad fplot(f,sInterval,'LineWidth',2); \\        
						    end    \\
						    legend('v=0.1','v=0.3','v=0.5','v=0.7','v=0.9');\\
						    set(gca,'fontsize',20);hold off    \\
						\end{tabular}
					}
					\end{center}
					\bigskip
					nu在此例為自由度的參數名稱,因此,由圖\ref{t}我們熟悉了T分配,當自由度越大						時,分配資料越往中心點趨近,而我們更有興趣在自由度為$30$或自由度更大時,T分						配和標準常態分配在圖形上的差異,而以MATLAB做圖形上驗證,如圖									\ref{t_normal}:
					\\
					\begin{figure}[H]	
		 		 		\centering	 			 	 
   				 		\includegraphics[width=1\textwidth]{\imgdir t_normal.jpg} 
   			 			\caption{T分配趨近標準常態分配)}   		
   			 			\label{t_normal}   			 		 
					\end{figure}
					\bigskip
					圖\ref{t_normal}中,紅色線段為標準常態分配,而藍色線段為T分配中,自由度由						$0.1$至$30$的變化,可看出當自由度越大時,分配圖形越接近標準常態分配的圖							形,也驗證統計學內所描述的內容。	
				}	
				\\
				\item{\textbf{\MJHmarker{卡方分配:}}\\
					\textbf{\color{darkblue}{卡方分布}\emph{(chi-square 										distribution)}},或寫作$x^2$分布是機率論與統計學中常用的一種機率分布。k個						獨立的標準常態分布變數的平方和服從自由度為k的卡方分布。卡方分布是一種特殊的						\textbf{\color{darkblue}{伽瑪分布}},是統計推斷中應用最為廣泛的機率分布之						一,例如\textbf{\color{darkblue}{假設檢驗}}和\textbf{\color{darkblue}						{信賴區間}}的計算。
					\bigskip
					\\
					令$X$為一連續隨機變數,若$X$符合卡方分布,且在自由度為k的情形下,其機率密度						分布函數(P.D.F)為:
					$$f_k(x)=\frac{\frac{1}{2}^{\frac{k}{2}}}{\gamma (\frac{k}{2})} 						x^{\frac{k}{2}-1} e^{\frac{-x}{2}} $$
					卡方分布的機率密度函數圖形:
					\begin{figure}[H]	
		 		 		\centering	 			 	 
   				 		\includegraphics[width=1\textwidth]{\imgdir chi2.jpg} 
   			 			\caption{卡方分配,自由度為5}   		
   			 			\label{chi2}   			 		 
					\end{figure}					
					圖\ref{chi2}可看出,卡方分配在正常的情況下,為一右偏機率密度函數,而繪製此						圖,利用MATLAB中的\textbf{chi2pdf}即可完成,如下所示:
					\begin{center}\colorbox{slight}{
						\begin{tabular}{p{0.9\textwidth}}
							\MJHmarker{\textbf{\color{darkblue}{MATLAB語法 :}}}\\		
							figure, hold on;    \\
    						title("卡方分配,自由度為5")\\
   							xInterval = [0 20];grid;\\   
   							f=@(x) chi2pdf(x,5);\\         
    						fplot(f,xInterval,'LineWidth',2.5,'color','blue'); \\
   							set(gca,'fontsize',20);hold off\\ 
						\end{tabular}
					}
					\end{center}
					\bigskip
					其中,我們知道卡方是有標準常態分配的平方轉換而成的分布,在數學上恆正,因此我						們只關心在X軸中,大於0的部分,而此例中,參數(自由度)設為$5$,為一特殊例子,						我們更有興趣在自由度不同時,此分配會有何表現,因此同樣利用MATLAB,實驗結果如						圖\ref{chi2_multi}:
					\begin{figure}[H]	
		 		 		\centering	 			 	 
   				 		\includegraphics[width=1\textwidth]{\imgdir chi2_multi.jpg} 
   			 			\caption{卡方分配,自由度(v)的變化}   		
   			 			\label{chi2_multi}   			 		 
					\end{figure}
					
					其中,程式語法如下:
					\begin{center}\colorbox{slight}{
						\begin{tabular}{p{0.9\textwidth}}
							\MJHmarker{\textbf{\color{darkblue}{MATLAB語法 :}}}\\		
							figure, hold on; \\
    						title("卡方分配,自由度(v)的變化")\\
    						nu=[5:5:30]; n=length(nu);\\
   							xInterval = [0 max(nu)*2];\\
   							for i=1:n\\
     						\quad	f=@(x) chi2pdf(x,nu(i));   \\      
     						\quad    fplot(f,xInterval,'LineWidth',2.5);  \\       
   							end \\
   							legend('v=5','v=10','v=15','v=20','v=25','v=30');\\
   							grid;        \\
   							set(gca,'fontsize',20); \\
   							hold off\\
						\end{tabular}
					}
					\end{center}
					由圖\ref{chi2_multi}可看出,卡方分配的自由度越大時,整個分配逐漸向右移動,						由右偏的機率密度函數,隨著自由度增加,逐漸趨近左偏的機率密度函數。
										
				}				
				\item{\textbf{\MJHmarker{F分配:}}\\
					F分布是美國統計學家為了彰顯英國統計學家費雪對統計的貢獻,以費雪名字開頭的字						母,當作這類型分布的名稱。以F分布為基礎,所衍生出的檢定方法,如									\textbf{\color{darkblue}{變方分析}}中的\textbf{\color{darkblue}{F檢						定}}及兩族群\textbf{\color{darkblue}{變方相等性檢定}}等,都是各領域的學						者經常使用的統計檢定。\\
					
					而此分布也是由上述中的卡方分布衍伸而來,在兩卡方分布中,各自除去自己的自由度						後再相除,即得到F分配,因此F分配和卡方相同皆恆正,而此特殊分布圖形如圖							\ref{f}:
					\begin{figure}[H]	
		 		 		\centering	 			 	 
   				 		\includegraphics[width=1\textwidth]{\imgdir f.jpg} 
   			 			\caption{F分配,自由度為(5,6)}   		
   			 			\label{f}   			 		 
					\end{figure}
					其中,程式語法如下:
					\begin{center}\colorbox{slight}{
						\begin{tabular}{p{0.9\textwidth}}
							\MJHmarker{\textbf{\color{darkblue}{MATLAB語法 :}}}\\		
							f = @(x) fpdf(x,5,6);\\
							fplot(f,[0,6],'LineWidth',3);\\
							ylim([0 1]);\\
							grid;  \\
							title("F Dist.");\\
							set(gca,'fontsize',20); \\
							legend;    \\
						\end{tabular}
					}
					\end{center}
					\bigskip
					我們知道,F分配藉由兩項自由度所控制,圖\ref{f}設定為$(5,6)$為一特殊例子,我						們更有興趣的是,當自由度有不同變化時,F分配會有何表現,因此,透過MATLAB實							驗,分成以下3種討論形式:
					\begin{enumerate}
						\item{自由度(1) $>$ 自由度(2)\\
							我們假設第一個自由度大於第二個自由度,並觀察當第一個自由度越大時,函								數圖形會有何種表現,如圖\ref{f1}:
							\begin{figure}[H]	
		 		 				\centering	 			 	 
   				 				\includegraphics[width=1\textwidth]{\imgdir f1.jpg} 
   			 					\caption{F分配,自由度(1)>自由度(2)}   		
   			 					\label{f1}   			 		 
							\end{figure}
							由圖\ref{f1}可觀察出,當第一項自由度越大時,分配會逐漸向座標軸的右								下所移動。							
						}
						\item{自由度(1) $=$ 自由度(2)\\
							我們假設第一個自由度等於第二個自由度,並另自由度由$5$至$30$,觀察兩								自由度相等,且同時變大時,函數圖形會有何種表現,如圖\ref{f2}:
							\begin{figure}[H]	
		 		 				\centering	 			 	 
   				 				\includegraphics[width=1\textwidth]{\imgdir f2.jpg} 
   			 					\caption{F分配,自由度(1)=自由度(2)}   		
   			 					\label{f2}   			 		 
							\end{figure}
							圖\ref{f2}可以明顯看出,函數圖形明顯趨於集中,且圖形分布逐漸向座標								軸右邊移動。
						}
						\item{自由度(1) $<$ 自由度(2)\\
							同理,我們假設第一個自由度小於第二個自由度,觀察函數圖形如圖										\ref{f3}:
							\begin{figure}[H]	
		 		 				\centering	 			 	 
   				 				\includegraphics[width=1\textwidth]{\imgdir f3.jpg} 
   			 					\caption{F分配,自由度(1)<自由度(2)}   		
   			 					\label{f3}   			 		 
							\end{figure}
							圖\ref{f3}中,隨著第二個自由度增大,函數呈現向座標軸右上移動。		
						}						
					\end{enumerate}
					
					綜合以上三種圖形,可發現F分配大致分布如圖\ref{f4}呈現之樣貌:
					\begin{figure}[H]	
		 		 		\centering	 			 	 
   				 		\includegraphics[width=1\textwidth]{\imgdir f4.jpg} 
   			 			\caption{F分配}   		
   			 			\label{f4}   			 		 
					\end{figure}					
					其中,程式語法如下:
					\begin{center}\colorbox{slight}{
						\begin{tabular}{p{0.9\textwidth}}
							\MJHmarker{\textbf{\color{darkblue}{MATLAB語法 :}}}\\		
							 figure,hold on; \\
							 v1 = [5:5:30];v2 = [5:5:30];\\							 
							 for i=1:length(v1)\\
   							 \quad	for j=1:length(v2)\\
      						\quad \quad  f = @(x) fpdf(x,v1(i),v2(j));\\
      						\quad \quad	  fplot(f,[0,2],'LineWidth',3,'color','b');\\
      						\quad \quad	  pause(0.3);\\
    						\quad	end\\
							 end\\
							 grid;ylim([0 1.5]);title("F Dist."); \\
							 set(gca,'fontsize',20);hold off;  \\ 
						\end{tabular}
					}
					\end{center}
				}
				\item{\textbf{\MJHmarker{貝塔分配:}}\\
					貝塔分配($\beta$ 分配),在函數圖形上,有各式各樣的風貌,包括左偏,右偏,甚						至均勻分配,其中改變其分配形狀的主要依據,就是來自參數$\alpha$ 以及$\beta$ 					,我們甚至也可以知道,當$\alpha$ 和$\beta$ 的大小關係改變,會造成圖形如何						的變化,因此我們和上述一樣區分成3種討論形式:
					\begin{enumerate}
						\item{$\alpha > \beta$\\
							我們假設$\alpha > \beta$,並觀察當$\beta$ 越來越接近$\alpha$ 								時,函數圖形會有何種表現,如圖\ref{beta1}:
							\begin{figure}[H]	
		 		 				\centering	 			 	 
   				 				\includegraphics[width=1\textwidth]{\imgdir beta1.jpg} 
   			 					\caption{$\beta$ 分配($\alpha > \beta$)}   		
   			 					\label{beta1}   			 		 
							\end{figure}
							由圖\ref{beta1}可觀察出,當$\beta$ 越趨近$\alpha$ 但同時不超過								$\alpha$ 時,分配中左偏傾向,會越來越不明顯,高峰逐漸向原點移動。					
						}
						\item{$\alpha = \beta$\\
							我們假設$\alpha = \beta$,並另兩者同時由$1$至$9$遞增,觀察兩								參數相等,且同時變大時,函數圖形會有何種表現,如圖\ref{beta2}:
							\begin{figure}[H]	
		 		 				\centering	 			 	 
   				 				\includegraphics[width=1\textwidth]{\imgdir beta2.jpg} 
   			 					\caption{$\beta$ 分配($\alpha = \beta$)}   		
   			 					\label{beta2}   			 		 
							\end{figure}
							圖\ref{beta2}可以明顯看出,當參數同時為$1$時,圖形形成均勻分配,								而參數同時遞增時,函數圖形明顯向中心點集中,並且對稱。
						}
						\item{$\alpha < \beta$\\%到這
							同理,我們假設$\alpha < \beta$,並且同時遞增$\beta$ 讓兩參數差距								逐漸拉大,觀察函數圖形如圖\ref{beta3}:
							\begin{figure}[H]	
		 		 				\centering	 			 	 
   				 				\includegraphics[width=1\textwidth]{\imgdir beta3.jpg} 
   			 					\caption{$\beta$ 分配($\alpha < \beta$)}   		
   			 					\label{beta3}   			 		 
							\end{figure}
							圖\ref{beta3}中,隨著$\beta$ 增大,函數逐漸右偏,高峰逐漸向原點移								動,偏態越來越明顯。		
						}		
						綜合以上三種圖形,我們可以觀察出,$\beta$分配中,當參數$\beta$愈大,且							離$\alpha$差距愈大,函數形狀愈為右偏,高峰愈往原點集中,如圖									\ref{beta3},反之,當$\alpha$愈大,且離$\beta$差距愈大時,函數愈為左							偏,此時高峰愈遠離原點,如圖\ref{beta1},而當此二參數相等時,函數								則是向中間集中,呈現對稱狀,如圖\ref{beta2},而我們同時將三種圖彙整後,							可得圖\ref{beta_final}:
						\begin{figure}[H]	
		 		 				\centering	 			 	 
   				 				\includegraphics[width=1\textwidth]{\imgdir 												beta_final.jpg} 
   			 					\caption{$\beta$ 分配}   		
   			 					\label{beta_final}   			 		 
						\end{figure}										
					\end{enumerate}
					其中,程式語法如下:
					\begin{center}\colorbox{slight}{
						\begin{tabular}{p{0.9\textwidth}}
							\MJHmarker{\textbf{\color{darkblue}{MATLAB語法 :}}}\\
       							 figure, hold on;\\
      							 alpha=1:2:9; beta=1:2:9;xInterval = [0 1];\\	        
      							 for i=1:length(alpha)\\
         						\quad	for u=1:length(beta)\\
            					\quad \quad		   f=@(x) betapdf(x,alpha(i),beta(u));       
             					\\ \quad \quad		   fplot(f,xInterval,'LineWidth',																3,'Color','b');\\                               
            					\quad	end\\
       							 end\\
       							 hold off; \\
        						 title("$\backslash beta$ 分配");ylim([0 4]);\\
       							 grid;set(gca,'fontsize',20);\\					 
						\end{tabular}
					}
					\end{center}
				}	
				\newpage					
				\item{\textbf{\MJHmarker{伽瑪分配:}}\\
					伽瑪分配,又稱Gamma分配,也是統計上極為常見的分配之一,包含本文提到的卡方分						配,或是指數分配,都是由Gamma分配的參數改變而得,而Gamma分配有兩種參數改變						其分配形狀:					
					\begin{enumerate}
						\item $\alpha$,\textbf{\color{darkblue}形狀參數(Shpae 										parameter)},影響P.D.F圖形之\textbf{\color{darkblue}陡峭},程								度。
						\item $\beta$,\textbf{\color{darkblue}尺度參數(Scale 										parameter)}影響P.D.F圖形之\textbf{\color{darkblue}散佈},程								度。
					\end{enumerate}
					由於Gamma分配一樣有兩個參數改變,我們同樣分開討論三種可能發生的情況:
					\begin{enumerate}
						\item {$\alpha > \beta$\\
							同樣我們假設$\alpha > \beta$,並觀察當$\beta$ 越來越接近$\alpha								$時,函數圖形會有何種表現,如圖\ref{gamma_case1}:
							\begin{figure}[H]	
		 		 				\centering	 			 	 
   				 				\includegraphics[width=1\textwidth]{\imgdir 												gamma_case1.jpg} 
   			 					\caption{Gamma 分配($\alpha > \beta$)}   		
   			 					\label{gamma_case1}   			 		 
							\end{figure}
							由圖\ref{gamma_case1}可觀察出,當控制陡峭程度的形狀參數$\alpha								$愈大於尺度參數$\beta$時,圖形和理論相同,愈為陡峭,而當$\beta$愈								接近$\alpha$時,圖形則欲趨近平緩。
						}
						\item {$\alpha = \beta$\\
							接著我們測試當$\alpha = \beta$時,並令兩者同時由$2$至$6$變化,觀								察函數圖形會有何種表現,如圖\ref{gamma_case2}:
							\begin{figure}[H]	
		 		 				\centering	 			 	 
   				 				\includegraphics[width=1\textwidth]{\imgdir 												gamma_case2.jpg} 
   			 					\caption{Gamma 分配($\alpha = \beta$)}   		
   			 					\label{gamma_case2}   			 		 
							\end{figure}
							由圖\ref{gamma_case2}可觀察出,雖然兩參數相等,但當兩者同時增大								時,函數一樣會由陡峭趨近平緩,而右偏傾向也愈來愈不明顯。
						}
						\item {$\alpha < \beta$\\
							最後我們測試當$\alpha < \beta$時,並讓$\beta$逐漸增大,遠離$									\alpha$,觀察函數圖形會有何種表現,如圖\ref{gamma_case3}:
							\begin{figure}[H]	
		 		 				\centering	 			 	 
   				 				\includegraphics[width=1\textwidth]{\imgdir 												gamma_case3.jpg} 
   			 					\caption{Gamma 分配($\alpha < \beta$)}   		
   			 					\label{gamma_case3}   			 		 
							\end{figure}
							由圖\ref{gamma_case3}可觀察出,圖形變化上與前兩者差異不大,同樣利								用$\beta$參數改變,亦可變化圖形至平緩。
						}
					\end{enumerate}
					由三種可能情況,我們同樣可以歸納出最終Gamma分配所有形狀的組合,大致呈現如圖						\ref{gamma_final}:
					\begin{figure}[H]	
		 		 		\centering	 			 	 
   				 		\includegraphics[width=1\textwidth]{\imgdir 												gamma_final.jpg} 
   			 			\caption{Gamma 分配}   		
   			 			\label{gamma_final}   			 		 
					\end{figure}
					有別於貝塔分配,Gamma分配無論參數大小變化如何,都不容易讓函數圖形由右偏至左						偏,而在此一樣整理出MATLAB語法提供參考:
					\begin{center}\colorbox{slight}{
						\begin{tabular}{p{0.9\textwidth}}
							\MJHmarker{\textbf{\color{darkblue}{MATLAB語法 :}}}\\		
							figure,hold on; \\
							v1 = [2:6];\\
							v2 = [2:6];\\
							for i=1:length(v1)\\
 							\quad   for j=1:length(v2)\\
     						\quad \quad	  f = @(x) gampdf(x,v1(i),v2(j));\\
      						\quad \quad	  fplot(f,[0,60],'LineWidth',3,'color','b');\\
  							\quad  end\\
							end\\
							grid; \\ 
							title("Gamma Dist.");  \\ 
							set(gca,'fontsize',20); \\
							hold off;\\
						\end{tabular}
					}
					\end{center}					
				}								
			\end{enumerate}
		\subsection{離散型函數}
			同理,在離散型函數在本文中亦可稱作離散型隨機變數,而日常生活中一樣有許多離散型隨機變數				的例子,最簡單的擲銅板就是在統計上典型的例子之一,而為了展現MATLAB在離散型函數圖形上				的發揮,我們一樣利用些例子加以說明:
			\begin{enumerate}
				\item{\textbf{\MJHmarker{二項分配:}}\\	
					在機率論和統計學中,\textbf{\color{darkblue}{二項分布}										(Binomial distribution)}是$n$個獨立的是/非試驗中成功的次數的離散機率分						布,	其中每次試驗的成功機率為$p$。這樣的單次成功/失敗試驗又稱為									\textbf{\color{darkblue}{伯努利試驗}}。實際上,當$n = 1$時,二項分布就						是伯努利分布,而在此我們利用簡單的直方圖來探討二項分配在圖形上呈現的樣貌:
					\begin{figure}[H]	
		 		 		\centering	 			 	 
   				 		\includegraphics[width=1\textwidth]{\imgdir bin.jpg} 
   			 			\caption{二項分配之直方圖}   		
   			 			\label{bin}   			 		 
					\end{figure}
					上述提到,二項分配是觀察$n$次試驗中,在成功機率為$p$之情形下的試驗結果,因此						我們得知,二項分配的圖形藉由兩個參數$n$、$p$來改變其分配樣貌,而	由圖								\ref{bin}可看出,在我們令$n$為$20$且$p$為$0.6$的情形下,成功次數發生在							$12$次的機率最高,而其中,$12$,即是二項分配在$n=20,p=0.6$中理論上的平均值						(期望值),藉由MATLAB實作圖形後,我們也能得確定實際與理論上並無差別。
					\\
					然而,在離散型分配上,我們有時候也關心其\textbf{\color{darkblue}{莖葉圖}						(Stem plot)},如圖\ref{bin_stem}:
					\begin{figure}[H]	
		 		 		\centering	 			 	 
   				 		\includegraphics[width=1\textwidth]{\imgdir bin_stem.jpg} 
   			 			\caption{二項分配之莖葉圖}   		
   			 			\label{bin_stem}   			 		 
					\end{figure}
					其MATLAB語法透過\textbf{stem}展現,而二項分配則是利用\textbf{binopdf}的						函數來實踐,程式碼如下:
					\begin{center}\colorbox{slight}{
						\begin{tabular}{p{0.9\textwidth}}
							\MJHmarker{\textbf{\color{darkblue}{MATLAB語法 :}}}\\		
							N=20;p=0.6;\\
							x=0:N;\\
							y=binopdf(x,N,p);\\
							fig=stem(x,y);\\
							fig.LineWidth=3;\\
							grid;  \\
							title("Bin. Dist.");   \\
							set(gca,'fontsize',20); \\
						\end{tabular}
					}
					\end{center}
					\bigskip
					最後,當我們觀察離散型分配函數時,次數相對成為重要的事情之一,而累積次數在特						定時候又是我們希望知道的事情,因此我們可以透過\textbf{stairs}呈現其階梯圖,						利用累積機率密度函數觀察離散分配會有何種表現,如圖\ref{bin_stairs}:
					\begin{figure}[H]	
		 		 		\centering	 			 	 
   				 		\includegraphics[width=1\textwidth]{\imgdir bin_stairs.jpg} 
   			 			\caption{二項分配之階梯圖}   		
   			 			\label{bin_stairs}   			 		 
					\end{figure}
					透過圖\ref{bin_stairs}可知,理論上會有將近8成的實驗,成功次數都在13次以							下,而成功次數超過15次的實驗,不超過1成,而呈現累積機率密度函數,與莖葉圖的語						法如下:
					\begin{center}\colorbox{slight}{
						\begin{tabular}{p{0.9\textwidth}}
							\MJHmarker{\textbf{\color{darkblue}{MATLAB語法 :}}}\\		
							N=20;p=0.6;\\
							x=0:N;\\
							y=binocdf(x,N,p);\\
							fig=stairs(x,y);\\
							fig.LineWidth=3;\\
							grid;  \\
							title("Bin. Dist. CDF"); \\  
							set(gca,'fontsize',20); \\
						\end{tabular}
					}
					\end{center}
					\bigskip
					將原本的\textbf{binopdf}改成\textbf{binocdf}即可得到累積機率密度函數,而						透過莖葉圖與階梯圖,我們更容易觀察出在離散型函數分配上,簡單的敘述統計量與其						分配特性等等。							
				}				
				\item{\textbf{\MJHmarker{Poisson 分配:}}\\	
					最後,我們在分配函數中,討論Poisson分配。常常在電視上聽到某某交通法規實施或						嚴格取締交通違規行動後,道路上每月發上車禍的次數明顯減少了。但是所謂的「明顯						減少」是怎麼判斷的呢,每個月都會偶爾發生一些意外事故,事件數量到底要有多大的						改變才算是明顯的變化?
					\bigskip
					\\					
					或者家裡附近的警察每兩個小時固定會出來巡邏一次,那麼家裡門前在兩個小時之內都						沒有任何警察經過的機率是多少呢?這些問題都可以仰賴Poisson 分配來解決。假設某						區域單位時間之內平均事件發生次數為$\lambda$,那麼在這區域中事件發生的次數$X						$就符合Poisson 分配。還有許多日常生活中週遭的現象也符合Poisson 分配,例							如:每小時進入學校大門口的人數、隔壁麵店每小時的客人數量、每次紅綠燈之間的車						流量等等。\footnote{資料來源:http://www.agron.ntu.edu.tw/biostat/P							oisson.html}	
					\bigskip
					\\					
					以下是Poisson分布的數學式:\\
					Poisson分布只有一個參數,單位時間平均事件發生次數$\lambda$。
					令$X$為一離散隨機變數,若$X$符合Poisson分布,其機率密度分布函數為(P.D.F)						為:
					
					$$f(x)=\frac{e^{-\lambda}\lambda^x}{x!}, \;\;  x=0,1,2,\cdots$$ 
					
					而其函數圖形如圖\ref{poisson_lam1}:
					\begin{figure}[H]	
		 		 		\centering	 			 	 
   				 		\includegraphics[width=1\textwidth]{\imgdir poisson_lam1.jpg} 
   			 			\caption{Poisson 分配之莖葉圖}   		
   			 			\label{poisson_lam1}   			 		 
					\end{figure}
					上述提到已知警察平均每兩個小時巡邏一次。那麼家裡門前在兩個小時內出現警察次數						($x$)的機率圖形如圖\ref{poisson_lam1},可簡單觀察出,2小時內出現
					0次警察的機率約為0.37左右,那在此題延伸,若是想知道出現1次以上,2次以上等等						呢?我們可以藉由累積機率密度函數輔佐解決此問題,如圖\ref{poisson_stairs}:
					\begin{figure}[H]	
		 		 		\centering	 			 	 
   				 		\includegraphics[width=1\textwidth]{\imgdir 												poisson_stairs.jpg} 
   			 			\caption{Poisson 分配之階梯圖}   		
   			 			\label{poisson_stairs}   			 		 
					\end{figure}
					由圖\ref{poisson_stairs}即可得知,警察出現超過1次的機率已經不到3成,而大						於2次的機率甚至不到1成,因此透過累積機率密度圖,亦可解決其他繁瑣問題。
					\bigskip
					\\					
					其中語法如下:
					\begin{center}\colorbox{slight}{
						\begin{tabular}{p{0.9\textwidth}}
							\MJHmarker{\textbf{\color{darkblue}{MATLAB語法 :}}}\\		
							x=[0:10];\\
							y=poisscdf(x,1);\\
							h=stairs(x,y);\\
							h.LineWidth=5;\\
							grid;  \\
							title("Poisson,$\backslash lambda$ = 1");   \\
							set(gca,'fontsize',20); \\
						\end{tabular}
					}
					\end{center}
											
					最後,整理出幾個在參數$\lambda$不同的情形下,Poisson的各種形狀表現,如圖						\ref{poisson_final}:
					\begin{figure}[H]	
		 		 		\centering	 			 	 
   				 		\includegraphics[width=1\textwidth]{\imgdir 												poisson_final.jpg} 
   			 			\caption{Poisson 分配之多種形態}   		
   			 			\label{poisson_final}   			 		 
					\end{figure}
					圖\ref{poisson_final}中,$\lambda$愈大,圖形愈趨近平緩,由於$\lambda$為						平均數,亦為變異數,因此隨著$\lambda$愈大高峰點也愈大,圖形也愈矮胖,而其語						法如下:
					
					\begin{center}\colorbox{slight}{
						\begin{tabular}{p{0.9\textwidth}}
							\MJHmarker{\textbf{\color{darkblue}{MATLAB語法 :}}}\\		
							x=[0:15];lam=[2:2:10];\\
							figure, hold on;\\
							for i=1:length(lam)\\
 							\quad  y=poisspdf(x,lam(i));\\
  							\quad  h=plot(x,y);\\
  							\quad  h.LineWidth=5;\\
							end\\
							grid;title("Poisson");  \\ 
							set(gca,'fontsize',20); \\
							legend(string("$\backslash lambda$ ="+lam));\\
							hold off;\\
						\end{tabular}
					}
					\end{center}					
				}	
			\end{enumerate}
	\section{亂數產生與相關圖形}
		亂數是一連串獨立數字的組成,在此章的亂數,卻也是有規律的一串數字,我們將討論到如何從特定分			配中抽一連串亂數,並做圖形分析,敘述統計量之討論,母體推論與驗證,樣本數大小的差異等等,而			利用MATLAB解決諸如此類的問題,而以下將舉幾個例子做討論:		
		\begin{enumerate}
		\item{\textbf{\MJHmarker{常態分配之亂數產生:}}\\		
			上節我們已經了解常態分配的圖形,以及其參數變化後,函數形狀會如何改變,而這節我們一樣透				過最常見的常態分配,隨機產生亂數,並進一步探討,而在產生亂數之前,我們釐清一個問題:				「亂數要產生多少個才夠?」,而這個問題也直接牽扯到\textbf{\color{darkblue}{樣本					數}}大小,我們從常態分配中產生亂數,要產生多少個亂數,這些樣本才會足夠像母體呢?以下我				們也利用程式解決此問題:
			\begin{figure}[H]
    		 	\centering
      			 \subfloat[N=30]{
       			 \includegraphics[scale=0.15]{\imgdir normal30.jpg}}
        		 \subfloat[N=100]{
       			 \includegraphics[scale=0.15]{\imgdir normal100.jpg}}
       			 \\
       			 \subfloat[N=300]{
       			 \includegraphics[scale=0.15]{\imgdir normal300.jpg}}
       			 \subfloat[N=1000]{
       			 \includegraphics[scale=0.15]{\imgdir normal1000.jpg}}
       			 \caption{樣本數大小之差異}   
   				 \label{normal_samplesize}
			\end{figure}			
			其中僅需改變N的大小,就能產生不同樣貌的常態分配樣本圖形,程式碼如下:			
			\begin{center}\colorbox{slight}{
				\begin{tabular}{p{0.9\textwidth}}
					\MJHmarker{\textbf{\color{darkblue}{MATLAB語法 :}}}\\		
					n=30;x=normrnd(0,1,1,n);\\					
					h=histogram(x);\\
					h.NumBins=30;h.FaceColor='red';\\					
					xlim([-3,3]);grid;set(gca,'fontsize',20);\\					
					title(string("Normal,n="+n));\\   					
				\end{tabular}
			}
			\end{center}		
			由圖\ref{normal_samplesize}可看出,當樣本數愈大時,圖形才會愈明顯像常態分佈,而若是				以qqplot圖來觀察的話,可以更明顯看出其中差異,如圖\ref{normal_qqplot}:
			\begin{figure}[H]
    		 	\centering
      			 \subfloat[N=30]{
       			 \includegraphics[scale=0.13]{\imgdir normal30.jpg}}
        		 \subfloat[qqplot,N=30]{
       			 \includegraphics[scale=0.13]{\imgdir normal_qq30.jpg}}
       			 \\
       			 \subfloat[N=300]{
       			 \includegraphics[scale=0.13]{\imgdir normal300.jpg}}
       			 \subfloat[qqplot,N=300]{
       			 \includegraphics[scale=0.13]{\imgdir normal_qq300.jpg}}
       			 \\
       			 \subfloat[N=1000]{
       			 \includegraphics[scale=0.13]{\imgdir normal1000.jpg}}   
       			 \subfloat[qqplot,N=1000]{
       			 \includegraphics[scale=0.13]{\imgdir normal_qq1000.jpg}}
       			 \caption{樣本數大小與qqplot比較}
   				 \label{normal_qqplot}
			\end{figure}
			圖\ref{normal_qqplot}我們已經能觀察出,樣本數超過300時已經夠像常態分配的形狀,當樣				本數1000時,qqplot幾乎吻合常態分配該有的樣貌。
			
			最後我們能夠透過boxplot來觀察我們產生亂數後的一些敘述統計量,亦能看出分配大至偏態形				狀等等,如圖\ref{normal_box};
			\begin{figure}[H]	
		 		 \centering	 			 	 
   				 \includegraphics[width=1\textwidth]{\imgdir normal_box.jpg} 
   			 	 \caption{常態分配}   		
   			 	 \label{normal_box}   			 		 
			\end{figure}
			由此也能看出,圖形大致呈現對稱狀,中位數落在$0$的位置左右,而也有幾個outlier在圖中,				其程式碼如下:
			\bigskip
			\begin{center}\colorbox{slight}{
				\begin{tabular}{p{0.9\textwidth}}
					\MJHmarker{\textbf{\color{darkblue}{MATLAB語法 :}}}\\		
					n=1000;\\
					x=normrnd(0,1,1,n);\\
					boxplot(x);\\
					grid;\\
					title(string("Normal,n="+n));  \\ 
					set(gca,'fontsize',20);\\
				\end{tabular}
			}
			\end{center}			
		}
		
		\item{\textbf{\MJHmarker{卡方分配之亂數產生:}}\\
			由上例可知,樣本數愈大,所產生的樣本分配愈趨近於母體,而卡方分配一樣如此,然而,在程式				繪圖中,仍有些參數設定須注意,例如\textbf{NumBins},若切割太少,則圖形顯示上不足以具				代表性,因此切割數也不能過低:如圖\ref{chi_hist_bin}:
			\begin{figure}[H]	
		 		 \centering	 			 	 
   				 \subfloat[bins=5]{
       			 \includegraphics[scale=0.15]{\imgdir chi_hist_b5.jpg}}
       			 \subfloat[bins=30]{
       			 \includegraphics[scale=0.15]{\imgdir chi_hist_b30.jpg}}
   			 	 \caption{卡方分配,bins大小差異}   		
   			 	 \label{chi_hist_bin}   			 		 
			\end{figure}
			而當我們最後決定好bins以及樣本數後,即可繪製卡方圖形,並且以理論的卡方分配線來配適此樣				本分配之直方圖,看兩者之間是否吻合,如圖\ref{chi_plot}:
			\begin{figure}[H]	
		 		 \centering	 			 	 
   				 \includegraphics[width=1\textwidth]{\imgdir chi_plot.jpg} 
   			 	 \caption{卡方分配與理論值之配適}   		
   			 	 \label{chi_plot}   			 		 
			\end{figure}			
			圖\ref{chi_plot}看出,理論的卡方分配(紅線)與我們隨機抽樣之直方圖,大致吻合,如此可以				簡單驗證此樣本確實是來自卡方分配。
			而透過理論線段重疊樣本圖形之程式碼如下:
			\begin{center}\colorbox{slight}{
				\begin{tabular}{p{0.9\textwidth}}
					\MJHmarker{\textbf{\color{darkblue}{MATLAB語法 :}}}\\		
					n=1000;\\
					x=chi2rnd(3,1,n);\\
					h=histogram(x,'Normalization','pdf');\\
					h.NumBins=30;\\					
					title("ChiSquare");   \\
					set(gca,'fontsize',20);\\
					hold on;\\
					f = @(x1) chi2pdf(x1,3);\\
					fplot(f,'LineWidth',3);\\
					hold off;\\
					xlim([0 15]);grid;\\
				\end{tabular}
			}
			\end{center}
			或是透過ECDF圖形,也能看出在累積機率密度函數中,理論值與我們抽樣值是否吻合,如圖					\ref{chi_ecdf}:
			\begin{figure}[H]	
		 		 \centering	 			 	 
   				 \includegraphics[width=1\textwidth]{\imgdir chi_ecdf.jpg} 
   			 	 \caption{卡方分配\_ ECDF圖}   		
   			 	 \label{chi_ecdf}   			 		 
			\end{figure}
			
			程式語法如下:
			\begin{center}\colorbox{slight}{
				\begin{tabular}{p{0.9\textwidth}}
					\MJHmarker{\textbf{\color{darkblue}{MATLAB語法 :}}}\\		
					figure ,hold on;\\
					n=1000;\\
					x=chi2rnd(3,1,n);\\
					h=cdfplot(x)'\\
					h.LineWidth=3;\\					
					title("ChiSquare");   \\
					set(gca,'fontsize',20);\\
					f = @(x1) chi2cdf(x1,3);\\
					fplot(f,'LineStyle',"--",'Color','r','LineWidth',4);\\
					hold off;grid;\\
				\end{tabular}
			}
			\end{center}
			圖\ref{chi_ecdf}中,紅線為理論值,藍線為樣本,可見大致吻合,也可確定此樣本是來自卡方				分配。			
			最後,我們已經大致確定此樣本真的是來自卡方分配後,以qqplot來觀察圖形會形成怎樣形狀,				驗證其不會符合常態分配之形狀,如圖\ref{chi_qq}:
			\begin{figure}[H]	
		 		 \centering	 			 	 
   				 \includegraphics[width=1\textwidth]{\imgdir chi_qq.jpg} 
   			 	 \caption{卡方分配\_ qqplot}   		
   			 	 \label{chi_qq}   			 		 
			\end{figure}
			圖\ref{chi_qq}明顯看出,樣本資料大致不落在$45$度線上,可見此分配確實不是常態分配。
		}
		\end{enumerate}	
	\section{抽樣分配}
		在上章節中提到透過某特定母體抽取亂數,做出多方面探討,然而,在日常生活中,有些情況卻無法直			接利用母體取得後之樣本,直接做進一步分析,中間反而需經過不同種類的函數變化,而衍生出新的樣			本,而此時我們更關心這樣新的樣本,會形成怎樣的分配,而此分配又可稱作抽樣分配,在此我們除				了討論抽樣分配外,也透過程式演練,驗證過去所學的理論,與如今實際操演的結果,是否雷同,因此			以下也舉幾個例子研究此問題:
		\begin{enumerate}
			\item{\textbf{\MJHmarker{中央極限定理}}\\
				中央極限定理是機率論中的一組定理。中央極限定理說明,在適當的條件下,大量相互獨立隨					機變數的平均數經適當標準化後依分布收斂於常態分布。這組定理是數理統計和誤差分析的理					論基礎,指出了大量隨機變數之抽樣分配近似服從常態分布的條件。				
				\\
				\bigskip
				為了實踐中央極限定理,我們假設從二項分配中抽取$n$組樣本,計算其平均數,並且重複實					驗1000次($N=1000$),觀察在不同數量之樣本底下,抽樣分配最後收斂的結果,如圖						\ref{clt_samplesize}:
				\begin{figure}[H]	
		 			 \centering	 			 	 
   					 \subfloat[n=5]{
       				 \includegraphics[scale=0.15]{\imgdir clt_n5.jpg}}
       				 \subfloat[n=10]{
       				 \includegraphics[scale=0.15]{\imgdir clt_n10.jpg}}
       				 \\
       				 \subfloat[n=30]{
       				 \includegraphics[scale=0.15]{\imgdir clt_n30.jpg}}
       				 \subfloat[n=100]{
       				 \includegraphics[scale=0.15]{\imgdir clt_n100.jpg}}
   			 		 \caption{CLT 樣本大小差異}   		
   			 		 \label{clt_samplesize}   			 		 
				\end{figure}
				由圖\ref{clt_samplesize}可看出,在$n=5$時圖形還有左偏傾向,但到了$n=30$圖形大					致已形成對稱的鐘形分布,而仔細觀察$x$座標,可以發現當$n$愈大,樣本愈集中,到了						$n=100$時,樣本大致已分布在$11.5$與$12.5$之間,可見樣本數愈大,變異程度愈小。
				\\				
				最後我們知道透過理論的常態分配曲線,來看是否此隨機樣本在經過平均的函數調整下,形成					的新的分配,真的服從常態分配:
				\begin{figure}[H]	
		 		 	\centering	 			 	 
   				 	\includegraphics[width=1\textwidth]{\imgdir clt_pf.jpg} 
   			 	 	\caption{抽樣分配趨近常態分配}   		
   			 		\label{clt_pf}   			 		 
				\end{figure}
				透過圖\ref{clt_pf},看出理論曲線,和我們的抽樣分配近乎吻合,可以推測此抽樣分配來					自常態分配,而也和過去所了解的中央極限定理相符,以下是透過MATLAB實作之程式碼:	
				\begin{center}\colorbox{slight}{
					\begin{tabular}{p{0.9\textwidth}}
						\MJHmarker{\textbf{\color{darkblue}{MATLAB語法 :}}}\\		
						figure,hold on;\\
						n=100;N=1000;\\						
						x=binornd(20,0.6,n,N);\\						
						histogram(mean(x),'NumBins',30,'Normalization','pdf');\\
						set(gca,'fontsize',20);\\
						f = @(x1) normpdf(x1,12,0.2);\\
						h=fplot(f);\\
						h.LineWidth=3;\\
						title("CLT n="+string(n));xlim([11 13]);grid;hold off;\\
					\end{tabular}
					}
				\end{center}				
				\bigskip
			}
			\newpage
			\item{\textbf{\MJHmarker{卡方分配的由來:}}\\
				卡方分配,是由標準常態分配經過函數轉換而來,這在數理統計中時常被提及,而這項真理如					何從理論實作,以下將以MATLAB用圖形實踐理論。同樣假設我們隨機從標準常態分配中抽樣					$n$筆資料,並對其進行平方轉換,得出新的資料集,在觀測新的資料集的分配是否服從卡方					分配:
				\begin{figure}[H]	
		 		 	\centering	 			 	 
   				 	\includegraphics[width=1\textwidth]{\imgdir chi2_1.jpg} 
   			 	 	\caption{$Z^2$之轉換 直方圖}   		
   			 		\label{chi2_1}   			 		 
				\end{figure}
				圖\ref{chi2_1}形狀類似理論上的卡方分配,且自由度為1,而我們這次透過ECDF圖來驗證					我們得出的樣本,是否真的和理論上一樣服從卡方1:
				\begin{figure}[H]	
		 		 	\centering	 			 	 
   				 	\includegraphics[width=1\textwidth]{\imgdir chi2_fromZ.jpg} 
   			 	 	\caption{$Z^2$之轉換 ECDF圖}   		
   			 		\label{chi2_fromZ}   			 		 
				\end{figure}
				圖\ref{chi2_fromZ}中,理論值為紅線,和樣本大致吻合,因此推測此抽樣分配最後形成					卡方分配,且自由度為1。
				而其中,程式碼如下:
				\begin{center}\colorbox{slight}{
					\begin{tabular}{p{0.9\textwidth}}
						\MJHmarker{\textbf{\color{darkblue}{MATLAB語法 :}}}\\		
						figure,hold on;\\
						n=1000;\\
						x=normrnd(0,1,1,n);\\
						newX=x.\^2;\\
						grid;xlim([0 10]);\\
						set(gca,'fontsize',20);\\
						h=cdfplot(newX);\\
						h.LineWidth=3;\\
						f = @(x1) chi2cdf(x1,1);\\
						fplot(f,'LineWidth',3,'LineStyle','- -');\\
						hold off;\\
					\end{tabular}
					}
				\end{center}	
			}
			\newpage
			\item{\textbf{\MJHmarker{分配可加性}}\\
				最後,我們討論到分配的可加性,分配可加性亦即兩特定分配相加後,依然服從該分配,例如					先前提及的卡方1加上卡方1,最後依然會形成卡方分配,且自由度為2,而如此數學理論,我					們依然能利用MATLAB實作其抽樣分配的樣貌:
				\begin{figure}[H]	
		 		 	\centering	 			 	 
   				 	\includegraphics[width=1\textwidth]{\imgdir chi2_2.jpg} 
   			 	 	\caption{卡方分配可加性}   		
   			 		\label{chi2_2}   			 		 
				\end{figure}
				我們透過隨機產生亂數,由卡方1中產生2組1000比亂數,接著對它進行相加的動作,結果顯					現如圖\ref{chi2_2},可以看出圖\ref{chi2_2}分配形狀接近卡方分配,但我們仍須一條					理論線來輔佐我們判斷,如圖\ref{chi2_2_plot};
				\begin{figure}[H]	
		 		 	\centering	 			 	 
   				 	\includegraphics[width=1\textwidth]{\imgdir chi2_2_plot.jpg} 
   			 	 	\caption{卡方分配可加性與理論線之配適}   		
   			 		\label{chi2_2_plot}   			 		 
				\end{figure}
				同樣透過理論值來輔佐,我們看出實際與理論極為接近,因此驗證卡方分配具有可加性,而					MATLAB實作語法如下:
				\begin{center}\colorbox{slight}{
					\begin{tabular}{p{0.9\textwidth}}
						\MJHmarker{\textbf{\color{darkblue}{MATLAB語法 :}}}\\		
						figure,hold on;\\
						n=1000;\\
						x1=chi2rnd(1,1,n);\\
						x2=chi2rnd(1,1,n);\\
						newX=x1+x2;\\
						histogram(newX,'Normalization','pdf');	\\
						grid;xlim=([0:10]);title("ChiSquare2");   \\
						set(gca,'fontsize',20);\\
						f = @(x3) chi2pdf(x3,2);\\
						fplot(f,'LineWidth',3,'LineStyle','--');	\\
						hold off;\\
						
					\end{tabular}
					}
				\end{center}
			}
		\end{enumerate}
	\section{結論}
		透過MATLAB,我們了解統計分配上的各種圖形呈現,不需要手動輸入繁瑣複雜的機率密度函數,僅須記			得短短幾行程式碼的語法,接著以各種圖形包括盒形圖,莖葉圖,直方圖等等呈現函數最原始的樣貌,			並且MATLAB中亦可以從某種特定分配中,產生一連串自訂樣本數的亂數,除了對於現今樣本數少有所益			處外,更能增加學習方式,而最後透過ECDF圖或是其他理論圖驗證此樣本來自的母體分配,對應到最初			數理統計學的多種理論,搭配實用的程式語言,交錯學習中更能以圖形記憶,增加基礎觀念。	
%\end{document}













